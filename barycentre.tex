\documentclass[12pt]{article}
\usepackage{stmaryrd}
\usepackage{graphicx}
\usepackage[utf8]{inputenc}
\usepackage[french]{babel}
\usepackage[T1]{fontenc}
\usepackage{hyperref}
\usepackage{verbatim}
\usepackage{color,soul}
\usepackage{amsmath}
\usepackage{amsfonts}
\usepackage{amssymb}
\usepackage{systeme}
\usepackage{tkz-tab}
\author{Destiné à la TerminaleS2\\Au Lycée de Dindéferlo}
\title{\textbf{Barycentre}}
\date{\today}
\usepackage{tikz}
\usetikzlibrary{arrows}
\usepackage[a4paper,left=20mm,right=20mm,top=15mm,bottom=15mm]{geometry}
\usepackage{mathtools}
\usepackage{systeme}

\DecimalMathComma

\begin{document}

\maketitle
\newpage

\section*{\underline{\textbf{\textcolor{red}{I.Barycentre de deux points:}}}}
\subsection*{\underline{\textbf{\textcolor{red}{2.Activité}}}}
\subsection*{\underline{\textbf{\textcolor{red}{3.Théorème et Définition}}}}
Soit A et B deux points du plan et $\alpha$ et $\beta$ deux nombres réels tels que $\alpha+\beta\neq0$. Il existe un et un seul point G, tel que $\alpha\vec{GA}+\alpha\vec{GB}=\vec{O}$.\\
Le point G est appelé :\\
$*$Barycentre des points A et B affectés des coefficicents $\alpha$ et $\beta$.\\
$*$Barycentre des points pondérés $(A,\alpha)$ et $(B,\beta)$\\
$*$Barycentre du système $(A,\alpha)$ et $(B,\beta)$
\subsection*{\underline{\textbf{\textcolor{red}{4.Démonstration}}}}
$\alpha\vec{GA}+\alpha\vec{GB}=\vec{O}$.\\
$\alpha\vec{AG}=\frac{\beta}{\alpha+\beta}\vec{AB}$.\\
$\alpha\vec{GA}+\alpha\vec{GB}=\vec{O}$.\\
$\alpha\vec{BG}=\frac{\alpha}{\alpha+\beta}\vec{BA}$.
\subsection*{\underline{\textbf{\textcolor{red}{Exemple}}}}
Déterminer dans chaque cas les réels $\alpha$ et $\beta$ pour que $G$ soit le barycentre des points pondérés $(A,\alpha)$ et $(B,\beta)$\\
a)$\vec{AB}=-\frac{2}{5}\vec{GB}$.\\
b)$3\vec{AG}=2\vec{BA}$.\\
\subsection*{\underline{\textbf{\textcolor{red}{5.Réduction de $\alpha\vec{MA}+\beta\vec{MB}$}}}}
\end{document}