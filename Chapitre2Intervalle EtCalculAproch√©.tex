\documentclass[12pt]{article}
\usepackage{stmaryrd}
\usepackage{graphicx}
\usepackage[utf8]{inputenc}

\usepackage[french]{babel}
\usepackage[T1]{fontenc}
\usepackage{hyperref}
\usepackage{verbatim}

\usepackage{color,soul}

\usepackage{amsmath}
\usepackage{amsfonts}
\usepackage{amssymb}
\usepackage{tkz-tab}
\author{\underline{Présentation Par:}\\Pathé BA}
\title{\textbf{Intervalle et calcul Approché}}
\date{\today}
\usepackage{tikz}
\usetikzlibrary{arrows}
%This command takes a colour as an optional argument; the default colour is black.

\newcommand{\myul}[2][black]{\setulcolor{#1}\ul{#2}\setulcolor{black}}
\begin{document}
\maketitle
\newpage
\begin{center}
\underline{\textbf{\textcolor{red}{4°)Intervalles et Distances}}}\\
\end{center}
Soit deux points d'abscisse $x$ et A d'abscisse a.\\
La distance entre $M$ et $A$ est notée:\\
$AM=d(x;a)$ et définie par  $d(x;a)=\mid x-a \mid $\\
$\blacktriangleright$Si $d(x;a)<b$ alors $x\in\left] a+b;a-b\right[ $

\begin{center}
\underline{\textbf{\textcolor{red}{Preuve}}}\\
\end{center}
$\blacktriangleright$Si $d(x;a)>b$ alors $x\in \left( \left] -\infty; a-b\right[ \cup \left] a+b;+\infty\right[\right)$\\

\begin{center}
\underline{\textbf{\textcolor{red}{Preuve}}}\\
\end{center}

\begin{center}
\underline{\textbf{\textcolor{red}{5°)Opération sur les intervalles:}}}\\
\end{center}

\begin{center}
\underline{\textbf{\textcolor{red}{a°)Centre-Rayon-Amplitude}}}\\
\end{center}

Soit un \textcolor{blue}{intervalle borné} de bornes a et b tels que a<b on a:\\
$\blacktriangleright$ Le centre de l'intervalle est : $c =\frac{a+b}{2}$\\
$\blacktriangleright$ Le rayon de l'intervalle est : $r =\frac{b-a}{2}$\\
$\blacktriangleright$ L'amplitude de l'intervalle est : $e=b-a$
\begin{center}
\underline{\textbf{\textcolor{red}{Exemple}}}\\
\end{center}
Dans chaque cas, donne le rayon, le centre et l'amplitude:\\
a°) $\left[-5;1\right] $ \\
b°) $\left] 2,6 \right] $ \\
\begin{center}
\underline{\textbf{\textcolor{red}{Attention}}}\\
\end{center}
Les intervalles ...... n'ont ni rayon ni centre ni amplitude\\
\begin{center}
\underline{\textbf{\textcolor{red}{Exercice d'application}}}\\
\end{center}
Recopie et complète ce tableau ci-dessous\\

\begin{tabular}{|c|c|c|c|}
  \hline
  Valeur absolue & Distance & Intervalle & Inégalité \\
  \hline
  $|x-10| \leqslant 9$ & & & \\
  \hline
  & $d(x;2) >3$ & & \\
  \hline
  &  &$x \in[4;6]$ & \\
  \hline
  & & & $x \leqslant -2$ ou $x \geqslant 2$ \\
  \hline
\end{tabular}

\begin{center}
\underline{\textbf{\textcolor{red}{b°)Réunion d'intervalles}}}\\
\end{center}
Soient I et J deux intervalles de $\mathbb{R}$. On appelle réunion des intervalle I et J, l'ensemble des éléments appartenant à I ou à J.\\
On note: $I \cup J$ 
\begin{center}
\underline{\textbf{\textcolor{red}{Exemple}}}\\
\end{center} 
Soit $A=\left[ -2,5\right] $ et $B=\left[ -3, +\infty\right[ $\\
Détermine $A \cup B$ 
\begin{center}
\underline{\textbf{\textcolor{red}{c°)Intersection d'intervalles}}}\\
\end{center}
On appelle intersection des intervalles I et J, l'ensemble des élément appartenant à la fois à I et J .\\
On note $I \cap J$
\begin{center}
\underline{\textbf{\textcolor{red}{Exemple}}}\\
\end{center} 
En Considérant l'exemple précédent, dtermine $A \cap B$
\begin{center}
\underline{\textbf{\textcolor{red}{V.Encadrement et Ordre}}}\\
\end{center}

\begin{center}
\underline{\textbf{\textcolor{red}{1.Définition}}}\\
\end{center}
Le nombre x est encadré par deux réels a et b lorsque :\\
$a\leqslant x \leqslant b$ \\
Cette inégalité est appelé encadrement du réel x à b-a près

\begin{center}
\underline{\textbf{\textcolor{red}{Exemple}}}\\
\end{center}
Encadrer à $10^{-3}$ près $\sqrt{3}$\\

\begin{center}
\underline{\textbf{\textcolor{red}{2°)Encadrement et opération}}}
\end{center}


\begin{center}
\underline{\textbf{\textcolor{red}{a°)Propriétés}}}
\end{center}

\begin{center}
\underline{\textbf{\textcolor{red}{b°)Encadrement d'une somme et d'une différence}}}\\
\end{center}

Soit $x$ et $y$ tels que $a<x<b$ et $c<y<d$ avec $a>0,b>0,c>0$ et $d>0$ on a:\\
\textbf{\textcolor{red}{$\circledast$ Encadrement de $x+y$}}\\
$a<x<b$\\
$+$\\
$c<y<d$\\
En faisant la somme membre à menbre, on a:\\
$a+c<x+y<b+d$\\
\textbf{\textcolor{red}{$\circledast$ Encadrement de $x-y$}}\\
On encadre d'abord $-y$.
ainsi, $c<y<d \Rightarrow -d<-y<-c$ \\
Donc \\
$a<x<b$\\
$-d<-y<-c$\\
En faisant la somme membre à menbre, on a:\\
$a-d<x-y<b-c$

\begin{center}
\underline{\textbf{\textcolor{red}{c°)Encadrement d'un produit et d'un quotient}}}\\
\end{center}
Soit deux réels $x$ et $y$ tels que $a<x<b$ et $c<y<d$ \\
\textbf{\textcolor{red}{$\circledast$ Encadrement de $x \times y$}}\\
$\blacktriangleright$ Si $a>0,b>0,c>0$ et $d>0$ alors,\\
$a<x<b$\\
$\times$\\
$c<y<d$\\
En faisant le produit membre à menbre, on a:\\
$a\times c <x\times y <b\times d$\\
\\
$\blacktriangleright$ Si $a>0,b>0,c<0$ et $d<0$ on a,\\
$c<y<d \Rightarrow -d<-y<-c$\\
donc\\
$a<x<b$\\
$\times$\\
$-d<-y<-c$\\
En faisant le produit membre à menbre, on a:\\
$-a\times d <-x\times y <-b\times c$\\
\textbf{\textcolor{red}{$\circledast$ Encadrement de $\frac{x}{y}$}}\\
Soit $a>0,b>0,c>0$ et $d>0$  \\
Dans ce cas, on encadre d'abord $\frac{1}{y}$ puis $x\times\frac{1}{y}$\\
Ainsi, on a $c<y<d \Rightarrow \frac{1}{d}<\frac{1}{y}<\frac{1}{c}$
\\
$\frac{1}{d}<\frac{1}{y}<\frac{1}{c}$\\
$\times$\\
$a<x<b$\\
Donc $\frac{a}{d}<\frac{x}{y}<\frac{b}{c}$\\
\begin{center}
\underline{\textbf{\textcolor{red}{Exercice d'application }}}\\
\end{center}

\begin{center}
\underline{\textbf{\textcolor{red}{Résolution}}}\\
\end{center}

\begin{center}
\underline{\textbf{\textcolor{red}{VI.Valeur approchée - notation scientifique}}}\\
\end{center} 
\begin{center}
\underline{\textbf{\textcolor{red}{1.Définitions }}}\\
\end{center}
$\blacktriangleright$On dit que a est une valeur approchée de x à $\varepsilon$ près si et seulement si 
$a-\varepsilon\leqslant x \leqslant a+\varepsilon$\\ c'est-à-dire $\mid x-a \mid \leqslant\varepsilon$\\
$\blacktriangleright$ On dit que a est une valeur approchée par défaut de $x$ à $\varepsilon$ près si, et seulement si,\\ $a \leqslant x \leqslant a+\varepsilon$ \\
\\
$\blacktriangleright$On dit que a est une valeur approchée par excès de x à $\varepsilon$ près si, et seulement si,\\
$a-\varepsilon\leqslant x \leqslant a$
\begin{center}
\underline{\textbf{\textcolor{red}{Exemple}}}\\
\end{center} 
1) Donner un encadrement de x dans les cas suivants :\\
a) 75,12 est une valeur approchée à $10^{-1}$ près de x\\
a) 75,12 est une valeur approchée à $10^{-2}$ près de x\\
a) 75,12 est une valeur approchée à $7.10^{-2}$ près de x\\
2) Donner un encadrement de $x-y$; $xy$ sachant que $3,23$ est une valeur approchée de 
$x$ à $0,1$ près et $-5\leqslant y \leqslant -2$.\\
\begin{center}
\underline{\textbf{\textcolor{red}{Solution}}}\\
\end{center}
1)\\a)
\\
$75,12-0,1\leq x \leq 75,12+0,1$\\
$75,02\leq x \leq 75,22$\\
\\
b)\\
$75,12-0,01\leq x \leq 75,12+0,01$\\
$75,\leq x \leq 75,13$\\
\\
c)\\
$75,12-0,07\leq x \leq 75,12+0,07$\\
$75,\leq x \leq 75,$\\
\\
2)\\
Puisque $3,23$ est une valeur approchée de 
$x$ à $0,1$ près, on a:\\
$3,23-0,1\leq x \leq 3,23+0,1$\\
\underline{\textcolor{red}{$3,13\leq x \leq 3,33$}}\\
\\
Encadrons -y, on a\\
$-5\leqslant y \leqslant -2 \Longrightarrow$ \underline{\textcolor{red}{$2\leqslant -y \leqslant 5$}}\\
\\
On a: \\
$75,12-0,1\leq x \leq 75,12+0,1$\\
$75,02\leq x \leq 75,22$\\
$\ast x-y$\\
\textcolor{red}{$\ast x-y$}\\
$3,13\leq x \leq 3,33$\\
+\\
$2\leqslant -y \leqslant 5$\\
$3,13+2\leq x-y \leq 3,33+5$\\
\textcolor{red}{$\ast x \times y$}\\
$3,13\leq x \leq 3,33$\\
$\times$\\
$2\leqslant -y \leqslant 5$\\
$3,13\times2 \leq -xy \leq 3,33\times5$\\
\\
\underline{\textbf{\textcolor{red}{Partie Entière}}}\\
$\forall x \in \mathbb{R}$  il existe un entier relatif ($b \in \mathbb{Z}$) et un seul tel que $b \leq x \leq b+1$\\ $b$  s'appelle la partie entière de $x$. On note $E(x)=b$\\
\underline{\textbf{\textcolor{red}{Approximation décimale }}}\\
Soit $x$ un réel, $n \in \mathbb{N}$, $b$ la partie entière de $x$.$10^{n}$ alors\\
$b.10^{-n} \leq x \leq (b+1)10^{-n}$\\
est un encadrement de $x$ à $10^{-n}$ près.\\
$b.10^{-n}$  est appelé approximation décimale de $x$.
\begin{center}
\underline{\textbf{\textcolor{red}{Exemple}}}\\
\end{center}
Soit 
$x=3,124124124124$, donner une approximation de $x$ à l'ordre $3$.
\begin{center}
\underline{\textbf{\textcolor{red}{Solution}}}\\
\end{center}
On a :\\
$ x.10^{3}=3124,124124124 \Longrightarrow E(x.10^{3})=3124 $\\
$ \Longrightarrow 3124 \leq x.10^{3} \leq 3125 $\\
$ \Longrightarrow 10^{-3}\times 3124 \leq x.10^{3} \times 10^{-3} \leq 3125 \times 10^{-3}$\\
$ \Longrightarrow 3,124 \leq x \leq 3,125 $\\
$3,124$ est donc une approximation décimale de $x$ à l'ordre $3$.

\begin{center}
\underline{\textbf{\textcolor{red}{Arrondi}}}\\
\end{center} 
Soit un nombre réel à plus de n chiffres après la virgule, l'arrondi d'ordre n de x est le décimal dont :\\
$\blacktriangleright$ Si le $n^{ième}$ chiffre est égal à lui-même si le $n^{ième}$ chiffre+1 $\in$ à $\left\lbrace 0;1;2;3;4 \right\rbrace $\\
$\blacktriangleright$ Si le $n^{ième}$ est ègal au $n^{ième}$ chiffre+1 si le $n^{ième}$ chiffre+1 $\in$ à $\left\lbrace 5;6;7;8;9 \right\rbrace $\\
\begin{center}
\underline{\textbf{\textcolor{red}{Exemple}}}\\
\end{center} 
Soit $x=4,02370956$\\
Donne l'arrondi d'ordre $0$ de $x$, l'arrondi d'ordre $2$ de $x$, l'arrondi d'ordre $3$ de $x$, l'arrondi d'ordre $5$ de $x$\\
\begin{center}
\underline{\textbf{\textcolor{red}{Solution}}}\\
\end{center}
l'arrondi d'ordre $0$ de $x$ : est $4$\\
l'arrondi d'ordre $2$ de $x$ : est $4,023$\\
l'arrondi d'ordre $3$ de $x$ : est $4,0237$\\
l'arrondi d'ordre $5$ de $x$ : est $4,02371$\\

\begin{center}
\underline{\textbf{\textcolor{red}{Notation scientifique}}}\\
\end{center} 
La notation scientifique d'un nombre réel $x$ est l'écriture de $x$ sous la forme 
$a.10^{p}$ où 
$1 \leq a \leq 10$ et $p \in \mathbb{Z}.$\\

\begin{center}
\underline{\textbf{\textcolor{red}{Exemple}}}\\
\end{center} 
On donne les écritures scientifiques de $254 ; 6000; 0,0064 ; 29,77$\\
On a : \\
$2543=2,543.10^{3} ; 6000=6.10^{3}$\\
$0,0064=6,4.10^{-3} ; 29,77=2,977.10^{1}$
\begin{center}
\underline{\textbf{\textcolor{red}{Solution}}}\\
\end{center}
\end{document}
