\documentclass[12pt]{article}
\usepackage{stmaryrd}
\usepackage{graphicx}
\usepackage[utf8]{inputenc}

\usepackage[french]{babel}
\usepackage[T1]{fontenc}
\usepackage{hyperref}
\usepackage{verbatim}

\usepackage{color,soul}

\usepackage{amsmath}
\usepackage{amsfonts}
\usepackage{amssymb}
\usepackage{systeme}
% \usepackage{tkz-tab} % Commenté car non utilisé dans ce document
\author{Destiné à la $2^{nde}S$\\Au Lycée de Dindéferlo}
\title{\textbf{Nombres Complexes}}
\date{\today}
\usepackage{tikz}
\usetikzlibrary{arrows}

\usepackage[a4paper,left=20mm,right=20mm,top=15mm,bottom=15mm]{geometry}
\usepackage{mathtools}
\usepackage{systeme}
\usetikzlibrary{arrows.meta}
\newcommand{\myul}[2][black]{\setulcolor{#1}\ul{#2}\setulcolor{black}}
\newcommand\tab[1][1cm]{\hspace*{#1}}

\DecimalMathComma
\begin{document}
\maketitle
\newpage

\section*{\textcolor{red}{\textbf{I. Notation de Vecteur}}}
\section*{\textcolor{red}{\textbf{1)Définition:}}}

Un vecteur est segment orienté representer par une ou deux lettres et surmonté d'une flêche.

\begin{tikzpicture}
  % Définir les points A et B
  \coordinate (A) at (0,0);
  \coordinate (B) at (2,3);

  % Dessiner le vecteur AB
  \draw[->, >=Stealth, thick] (A) -- (B) node[midway, above] {$\vec{u}$};
  \draw[->, >=Stealth, thick] (A) -- (B) node[midway, below] {$\vec{AB}$};
  % Marquer les points A et B
  \node[below left] at (A) {$A$};
  \node[above right] at (B) {$B$};
\end{tikzpicture}

\section*{\textcolor{red}{\textbf{2)Caractéristique d'un vecteur:}}}

Un vecteur est caractérisé par:\\

\begin{itemize}
  \item \textbf{Sa Direction :} c'est la droite qui supporte le vecteur : (AB).
  \item \textbf{Son Sens :} de l'origine vers l'extrémité : de A vers B.
  \item \textbf{Sa Norme :} c'est la distance AB.
\end{itemize}

\section*{\textcolor{red}{\textbf{Remarque:}}} 
\section*{\textcolor{red}{\textbf{2)Caractéristique d'un vecteur:}}}
\begin{itemize}
  \item Toute droite parallèle à la direction d'un vecteur est également considérée comme sa direction. Ainsi, dans ce cas, on dit qu'un vecteur admet une infinité de directions.
  \item Lorsque l'origine et l'extrémité d'un vecteur coïncident, on dit que le vecteur est nul, et on le note par $\vec{O}$.
\end{itemize}
\section*{\textcolor{red}{\textbf{3)Egalité vectorielle:}}}
\section*{\textcolor{red}{\textbf{a)Définition:}}}
On dit que deux vecteurs sont égaux lorsqu'ils ont la même direction, le même sens et la même norme.
\section*{\textcolor{red}{\textbf{Exemple:}}} 
ABCD est un parallélogramme de centre O. I et J les milieux respectifs des [AB] et [CD]\\
1)Détermine tous les vecteurs égaux à $\vec{BI}$\\
2)Donner une direction de $\vec{BI}$\\
3)Détermine la direction, le sens et la norme de $\vec{DO}$\\
\section*{\textcolor{red}{\textbf{Solution:}}}

\begin{tikzpicture}
  % Définir les sommets du parallélogramme
  \coordinate (A) at (0,0);
  \coordinate (B) at (2,0);
  \coordinate (C) at (3,2);
  \coordinate (D) at (1,2);

  % Dessiner les côtés du parallélogramme
  \draw (A) -- (B) -- (C) -- (D) -- cycle;

  % Marquer les sommets A, B, C, D
  \foreach \point in {A, B, C, D} {
    \node[circle, fill, inner sep=1.5pt, label=below:\textcolor{blue}{$\point$}] at (\point) {};
  }
\end{tikzpicture}
\section*{\textcolor{red}{\textbf{b)Propriété relative au parallélogramme:}}}
ABCD est un parallélogramme si et seulement si l'une des relations suivantes est vérifiée :

\begin{enumerate}
  \item $\vec{AB} = \vec{DC}$.
  \item $\vec{AD} = \vec{BC}$.
  \item Les droites (AB) et (DC) sont parallèles, et les droites (AD) et (BC) sont parallèles.
  \item Les segments [BD] et [AC] se coupent en leur milieu.
\end{enumerate}
\section*{\textcolor{red}{\textbf{II)Addition de deux vecteurs:}}}
Soit deux vecteur non nuls $\vec{u}$ et $\vec{v}$, tels que $\vec{u}=\vec{AB}$ et $\vec{v}=\vec{CD}$ Construisons.\\ $\vec{u}+\vec{v}$\\
\\
\begin{tikzpicture}
  % Dessiner les vecteurs
  \draw[->, >=Stealth, thick] (0,0) -- (2,1) node[midway, below left] {$\vec{u}$};
  \draw[->, >=Stealth, thick] (0,2) -- (2,2) node[midway, below right] {$\vec{v}$};
\end{tikzpicture}

\section*{\textcolor{red}{\textbf{1)La Relation de Chasles:}}}

La relation de Chasles est une propriété fondamentale des vecteurs. Elle énonce que pour trois points non alignés \(A\), \(B\), et \(C\), le vecteur \(\vec{AB}\) suivi du vecteur \(\vec{BC}\) est équivalent au vecteur \(\vec{AC}\). Mathématiquement, cela s'exprime comme suit :

\[
\vec{AB} + \vec{BC} = \vec{AC}
\]

Cette relation peut également être visualisée graphiquement.

\subsection*{Visualisation Graphique}

[[Considérons trois points non alignés \(A\), \(B\), et \(C\). La flèche \(\vec{AB}\) représente le vecteur allant de \(A\) à \(B\), et la flèche \(\vec{BC}\) représente le vecteur allant de \(B\) à \(C\). En plaçant la queue de \(\vec{BC}\) à la tête de \(\vec{AB}\), on obtient une flèche qui représente le vecteur \(\vec{AC}\).]]

\begin{center}
\begin{tikzpicture}
  % Définir les coordonnées des points
  \coordinate (A) at (0,0);
  \coordinate (B) at (2,1);
  \coordinate (C) at (4,3);

  % Dessiner les vecteurs
  \draw[->, >=Stealth, thick] (A) -- (B) node[midway, below left] {$\vec{AB}$};
  \draw[->, >=Stealth, thick] (B) -- (C) node[midway, below right] {$\vec{BC}$};
  \draw[->, >=Stealth, thick] (A) -- (C) node[midway, above] {$\vec{AC}$};
\end{tikzpicture}
\end{center}

\subsection*{Propriétés de la Relation de Chasles}

La relation de Chasles possède plusieurs propriétés intéressantes :

\begin{itemize}
  \item \textbf{Additivité :} Pour tout point \(D\), \(\vec{AB} + \vec{BC} + \vec{CD} = \vec{AD}\).
  \item \textbf{Inversion :} \(\vec{AB} = -\vec{BA}\).
  \item \textbf{Annulation :} Si \(A = B\), alors \(\vec{AB} = \vec{0}\), le vecteur nul.
\end{itemize}

[[La relation de Chasles est un outil puissant dans la manipulation des vecteurs et joue un rôle clé dans divers domaines des mathématiques et de la physique.]]
\section*{\textcolor{red}{\textbf{2)Somme de deux vecteurs:}}}
Soient $\vec{u}$ et $\vec{v}$ deux vecteurs du plan.
\subsection*{Cas 1 : Même Origine}

\textcolor{yellow}{Dans le cas où les vecteurs ont la même origine, la somme vectorielle est simplement la flèche reliant l'origine au point final du deuxième vecteur.}
\begin{center}
\begin{tikzpicture}[scale=2]
  % Vecteur 1 (en rouge)
  \draw[red, thick, ->] (0,0) -- (1.5,1) node[midway, above left] {$\mathbf{v}_1$};
  
  % Vecteur 2 (en bleu)
  \draw[blue, thick, ->] (0,0) -- (-1,1.5) node[midway, below right] {$\mathbf{v}_2$};
\end{tikzpicture}
\end{center}

\begin{center}
\begin{tikzpicture}[scale=2]
  % Vecteur (en rouge)
  \draw[red, thick, ->] (0,0) -- (1.5,1) node[midway, above left] {$\mathbf{v}$};
\end{tikzpicture}
\end{center}

\subsection*{Cas 2 : Origines Différentes}

\textcolor{yellow}{Lorsque les vecteurs ont des origines différentes, la somme vectorielle est obtenue en les plaçant bout à bout, c'est-à-dire en ajoutant la tête du premier vecteur à la queue du deuxième vecteur.}
\begin{center}
\begin{tikzpicture}[scale=2]
  % Vecteur original (en rouge)
  \draw[red, thick, ->] (0,0) -- (1.5,) node[midway, above left] {$\mathbf{v}$};
  
  % Vecteur représentatif (en bleu)
  \draw[blue, thick, ->] (1.5,1) -- (3,2) node[midway, above left] {$\mathbf{v}$};
\end{tikzpicture}
Compilation rapide
\end{center}
\begin{center}
\begin{tikzpicture}[scale=2]
  % Vecteur u (en rouge)
  \draw[red, thick, ->] (0,0) -- (1.5,1) node[midway, above left] {$\mathbf{u}$};
  
  % Vecteur v (en bleu)
  \draw[blue, thick, ->] (0,0) -- (-1.9,-2) node[midway, below right] {$\mathbf{v}$};
  
  % Annotation v+u sur le vecteur v
  \node[above left] at (-0.75,-0.5) {$\vec{v} + \vec{u}$};
\end{tikzpicture}
\end{center}
\section*{\textcolor{red}{\textbf{II.Construire une combinaison linéaire de vecteurs.}}}
\section*{\textcolor{red}{\textbf{2.1. Décomposer un vecteur à l’aide de la relation de Chasles}}}

\textcolor{red}{\textbf{2.2 Définition}}

\textcolor{red}{\textbf{2.3 Exemple}}

\textcolor{red}{\textbf{2.4 Propriétés}}

\textcolor{red}{\textbf{Exemple}}

\section*{\textcolor{red}{\textbf{III. Passer de la relation vectorielle $\vec{AC}=\lambda \vec{AB}$ à la relation algébrique $\vec{AC}=|\lambda|\vec{AB}$}}}
\section*{\textcolor{red}{\textbf{IV. Utiliser les relations vectorielles pour démontrer des propriétés géométriques
( distance, alignement, milieu, parallélisme)}}}
\end{document}
