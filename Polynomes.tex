\documentclass[12pt]{article}
\usepackage{stmaryrd}
\usepackage{graphicx}
\usepackage[utf8]{inputenc}

\usepackage[french]{babel}
\usepackage[T1]{fontenc}
\usepackage{hyperref}
\usepackage{verbatim}

\usepackage{color,soul}

\usepackage{amsmath}
\usepackage{amsfonts}
\usepackage{amssymb}
\usepackage{systeme}
% \usepackage{tkz-tab} % Commenté car non utilisé dans ce document
\author{Destiné à la $2^{nd}$S\\Au Lycée de Dindéferlo}
\title{\textbf{Polynomes}}
\date{\today}
\usepackage{tikz}
\usetikzlibrary{arrows}

\usepackage[a4paper,left=20mm,right=20mm,top=15mm,bottom=15mm]{geometry}
\usepackage{mathtools}
\usepackage{systeme}
\usetikzlibrary{arrows.meta}
\newcommand{\myul}[2][black]{\setulcolor{#1}\ul{#2}\setulcolor{black}}
\newcommand\tab[1][1cm]{\hspace*{#1}}

\DecimalMathComma
\begin{document}
\maketitle
\newpage

\section*{\underline{\textcolor{red}{\textbf{I. Généralités }}}}

\subsection*{\underline{\textcolor{red}{\textbf{1. Monômes}}}}

\subsection*{\underline{\textcolor{red}{\textbf{a. Définition et vocabulaire}}}}

\subsection*{\underline{\textcolor{red}{\textbf{b. Exemples}}}}

\subsection*{\underline{\textcolor{red}{\textbf{c. Remarque}}}}

\subsection*{\underline{\textcolor{red}{\textbf{2. Polynômes}}}}

\subsection*{\underline{\textcolor{red}{\textbf{a. Définition}}}}

\subsection*{\underline{\textcolor{red}{\textbf{b. Exemple}}}}

\subsection*{\underline{\textcolor{red}{\textbf{c. Remarque}}}}

\subsection*{\underline{\textcolor{red}{\textbf{II. Trinômes du second degré}}}}

\subsection*{\underline{\textcolor{red}{\textbf{1. Définition et exemple}}}}

\subsection*{\underline{\textcolor{red}{\textbf{2. Factorisation de $ax^{2}+bx+c$}}}}
\end{document}