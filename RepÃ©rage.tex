\documentclass[12pt]{article}
\usepackage{stmaryrd}
\usepackage{graphicx}
\usepackage[utf8]{inputenc}

\usepackage[french]{babel}
\usepackage[T1]{fontenc}
\usepackage{hyperref}
\usepackage{verbatim}

\usepackage{color,soul}

\usepackage{amsmath}
\usepackage{amsfonts}
\usepackage{amssymb}
\usepackage{systeme}
% \usepackage{tkz-tab} % Commenté car non utilisé dans ce document
\author{Destiné à la $2^{nde}S$\\Au Lycée de Dindéferlo}
\title{\textbf{Chapitre III: Repère cartésien}}
\date{\today}
\usepackage{tikz}
\usetikzlibrary{arrows}

\usepackage[a4paper,left=20mm,right=20mm,top=15mm,bottom=15mm]{geometry}
\usepackage{mathtools}
\usepackage{systeme}
\usetikzlibrary{arrows.meta}
\newcommand{\myul}[2][black]{\setulcolor{#1}\ul{#2}\setulcolor{black}}
\newcommand\tab[1][1cm]{\hspace*{#1}}

\DecimalMathComma
\begin{document}
\maketitle
\newpage

\section*{\underline{\textcolor{red}{\textbf{I. Repères}}}}
\subsection*{\underline{\textcolor{red}{\textbf{1.Repère de la droite}}}}
\subsection*{\underline{\textcolor{red}{\textbf{2.Définition}}}}
On appelle repère de la droite la donnée d'une droite $(\mathfrak{D})$, d'un point O appartenant à la droite $(\mathfrak{D})$ et d'un vecteur $\vec{i}$ non nul ayant pour direction la droite $(\mathfrak{D})$\\
\definecolor{ffqqtt}{rgb}{1,0,0.2}
\definecolor{xdxdff}{rgb}{0.49019607843137253,0.49019607843137253,1}
\begin{tikzpicture}[line cap=round,line join=round,>=triangle 45,x=1cm,y=1cm]
\clip(-9.546585365853664,-5.016536585365853) rectangle (7.098292682926826,4.7342926829268315);
\draw [line width=2pt] (-5.65,3.35)-- (1.97,-1.49);
\draw [->,line width=2pt,color=ffqqtt] (-2.0957454902442016,1.092442017425451) -- (-2.933468597373911,1.6245391090931405);
\begin{scriptsize}
\draw[color=black] (-0.12629268292683377,0.5199512195121967) node {$D$};
\draw [fill=xdxdff] (-2.0957454902442016,1.092442017425451) circle (2.5pt);
\draw[color=xdxdff] (-2.4164390243902485,0.7088292682926846) node {$O$};
\end{scriptsize}
\end{tikzpicture}

La repére de la droite $(\mathfrak{D})$ est noté: $\mathfrak{D}(O;\vec{i})$

Avec O l'origine du repère et le vecteur $\vec{i}$ la base de ce repère
$\mathfrak{D}(O;\vec{i})$
\subsection*{\underline{\textcolor{red}{\textbf{a.Abscisse:}}}}
Soit un repère de la droite $\mathfrak{D}(O;\vec{i})$ pour tout point M de la droite $(\mathfrak{D})$, il existe un et un seul réel $x$ tel que $\vec{OM}=x\vec{i}$
Le réel $x$ est 	appelé alors \textcolor{red}{abscisse} du point M dans le repère $\mathfrak{D}(O;\vec{i})$ noté $x_{M}$

\underline{\textcolor{red}{\textbf{Exemple}}}

\definecolor{ffqqtt}{rgb}{1,0,0.2}
\definecolor{xdxdff}{rgb}{0.49019607843137253,0.49019607843137253,1}
\begin{tikzpicture}[line cap=round,line join=round,>=triangle 45,x=1cm,y=1cm]
\clip(-9.95815063414635,-6.114224975609752) rectangle (10.524965463414636,5.68427843902439);
\draw [line width=2pt] (-5.65,3.35)-- (1.97,-1.49);
\draw [->,line width=2pt,color=ffqqtt] (-2.0957454902442016,1.092442017425451) -- (-2.933468597373911,1.6245391090931405);
\begin{scriptsize}
\draw[color=black] (0.29769131707316904,1.1562813658536601) node {$D$};
\draw [fill=xdxdff] (-2.0957454902442016,1.092442017425451) circle (2.5pt);
\draw[color=xdxdff] (-2.5019535609756134,0.6134930731707335) node {$O$};
\draw [fill=xdxdff] (-3.6966339387215585,2.109279299660413) circle (2.5pt);
\draw[color=xdxdff] (-3.4732589268292724,2.727510634146343) node {$C$};
\draw [fill=xdxdff] (-4.967225290138013,2.916321575363252) circle (2.5pt);
\draw[color=xdxdff] (-4.730242341463419,3.5274091707317083) node {$E$};
\draw [fill=xdxdff] (-1.2071681826623075,0.5280438325571613) circle (2.5pt);
\draw[color=xdxdff] (-0.987859902439027,1.1562813658536601) node {$F$};
\draw [fill=xdxdff] (-0.16406728513245916,-0.13450319422032775) circle (2.5pt);
\draw[color=xdxdff] (0.06914887804877864,0.4706540487804897) node {$G$};
\end{scriptsize}
\end{tikzpicture}

Déterminer les abscises des points $O$, $C$, $E$, $F$, $G$ sachant que la droite est regulièrement graduée
\subsection*{\underline{\textcolor{red}{\textbf{Solution}}}}
$x_{D}=?$
\subsection*{\underline{\textcolor{red}{\textbf{c.Mesure algébrique}}}}
Soit $(\mathfrak{D})$ une droite munie d'un repère $(O;\vec{i})$ et A et B deux points de la droite $(\mathfrak{D})$.

On appelle mesure algébrique du bipoint $(A;B)$ ou du vecteur $\vec{AB}$, le réel noté par: 
$\overline{AB}$ et défini par : $\overline{AB}=x_{B}-x_{A}$ 
\subsection*{\underline{\textcolor{red}{\textbf{Exemple}}}}
Soit $(\mathfrak{D})$ une droite  munie d'un repère $(O;\vec{i})$

1)placer les points A,B,C,D et E de $(\mathfrak{D})$ tel que 

$x_{A}=-1,5$; $x_{B}=3$; $x_{C}=1$; $x_{D}=-2$; $x_{D}=-1,5$ et $x_{E}=2,5$

2)Calculer $\overline{AD}$; $\overline{DE}$; $\overline{CA}$; $\overline{DA}$; $\overline{BD}$

3)Placer les points F et G tel que $\overline{DF}=-3$ et $\overline{GE}=3,5$
\subsection*{\underline{\textcolor{red}{\textbf{d.Milieux d'un segment}}}}
Soit $(\mathfrak{D})$ une droite munie d'un repère $(O,\vec{i})$

Soit I le milieu de [AB] alors $x_{I}=\frac{x_{A}+x{B}}{2}$

\subsection*{\underline{\textcolor{red}{\textbf{Exemple}}}}
Soit $\mathfrak{D}(O;\vec{i})$ un repère et $A$ et $B$ deux points de $(\mathfrak{D})$ tel que $x_{A}=3$ et $x_{B}=-5$

Déterminer l'abscisse du point $I$, milieu de $[AB]$
\subsection*{\underline{\textcolor{red}{\textbf{e.Relation de Chasles}}}}
Soit une droite $(\mathfrak{D})$ une droite munie d'un repère $(O,\vec{i})$.

Pour tout point $A$ ; $B$ et $C$ de $(\mathfrak{D})$ 

On a: $\overline{AB}+\overline{BC}=\overline{AC}$
\subsection*{\underline{\textcolor{red}{\textbf{Preuve}}}}
On sait que: \[
    \begin{cases}
        \overline{AB} &= x_{B}-x_{A} \\
        \overline{BC} &= x_{C}-x_{B}
    \end{cases}
    \]
alors $\overline{AB}+\overline{BC}=x_{B}-x_{A}+x_{C}-x_{B}$
$=x_{C}-x_{A}$

$\overline{AB}+\overline{BC}=\overline{AC}$
\subsection*{\underline{\textcolor{red}{\textbf{Remarque}}}}
\begin{itemize}
    \item[$\ast $] $\overline{-AB}=\overline{BA}$
    \item[$\ast $] $\overline{AB}=\left\lVert \overline{BA} \right\rVert $
    \item[$\ast $] si $\overline{AB}=0$ alors $x_{A}=x_{B}$ ou A=B
    \item[$\ast $] $AB=\overline{AB}=\overline{BA}$
\end{itemize}
\subsection*{\underline{\textcolor{red}{\textbf{2.Repère du plan}}}}
\subsection*{\underline{\textcolor{red}{\textbf{a.Définition}}}}
On appelle repère du plan, la donnée de deux repères de droites ayant la même origine et de base non colinéaires.

Les repères $\mathfrak{D}(O,\vec{i})$ et $\Delta(O,\vec{i})$ forment un repère du plan noté: $(O;\vec{i};\vec{j})$

avec le point O origine du repère et le couple de vecteur $(\vec{i};\vec{j})$ base de ce repère.

\definecolor{uuuuuu}{rgb}{0.26666666666666666,0.26666666666666666,0.26666666666666666}
\definecolor{ffqqqq}{rgb}{1,0,0}
\definecolor{qqqqcc}{rgb}{0,0,0.8}
\begin{tikzpicture}[line cap=round,line join=round,>=triangle 45,x=1cm,y=1cm]
\clip(-7.82,-3.28) rectangle (9.22,5.68);
\draw [line width=2pt,color=qqqqcc] (-4.04,4.02)-- (1.92,-0.42);
\draw [line width=2pt,color=ffqqqq] (-3.94,-0.2)-- (1.86,3.84);
\draw [->,line width=2pt] (-1.0642101207295147,1.8031363986642692) -- (-1.8331840565436535,2.375996176351312);
\draw [->,line width=2pt] (-1.0642101207295147,1.8031363986642692) -- (-0.36352526740536684,2.2911996413245372);
\begin{scriptsize}
\draw[color=qqqqcc] (-1.76,3.05) node {$\Delta$};
\draw[color=ffqqqq] (0.16,3.09) node {$D$};
\draw [fill=uuuuuu] (-1.0642101207295147,1.8031363986642692) circle (2pt);
\draw[color=uuuuuu] (-1.08,1.47) node {$O$};
\draw[color=black] (-1.7,1.97) node {$j$};
\draw[color=black] (-0.48,2.01) node {$i$};
\end{scriptsize}
\end{tikzpicture}

Soit le repère du plan $(O;\vec{i};\vec{j})$ pour tout point M du plan, il existe un unique couple de nature $(x;y)$ tel que $\overrightarrow{OM}=x_{i}+x_{j}$
\begin{itemize}
    \item Les réels $x$ et $y$ sont appelés coordonnées du plan $M$ dans le repère $(O;\vec{i};\vec{j})$
    \item Le réel $x$ est l'abscises de $M$ dans $(O;\vec{i};\vec{j})$ et se note: $x_{M}$.
    \item Le réel $y$ est l'ordonné de M dans $(O;\vec{i};\vec{j})$ et se note par: $y_{M}$
    
    par conséquent: les coordonnées du point $M$ peuvent se noté par: $x_{M}; y_{M}$
\end{itemize}
\subsection*{\underline{\textcolor{red}{\textbf{Exemple}}}}
Placer les points $A(-1,2)$, $B(3,2)$, $C   (0,4)$ dans le repère $(O;\vec{i};\vec{j})$

\definecolor{qqqqcc}{rgb}{0,0,0.8}
\begin{tikzpicture}[line cap=round,line join=round,>=triangle 45,x=1cm,y=1cm]
\clip(-7.82,-3.28) rectangle (9.22,5.68);
\draw [line width=2pt,color=qqqqcc] (-4.04,4.02)-- (1.92,-0.42);
\draw [line width=2pt,color=ffqqqq] (-3.94,-0.2)-- (1.86,3.84);
\draw [->,line width=2pt] (-1.0642101207295147,1.8031363986642692) -- (-1.8331840565436535,2.375996176351312);
\draw [->,line width=2pt] (-1.0642101207295147,1.8031363986642692) -- (-0.36352526740536684,2.2911996413245372);
\begin{scriptsize}
\draw[color=qqqqcc] (-1.76,3.05) node {$\Delta$};
\draw[color=ffqqqq] (0.16,3.09) node {$D$};
\draw [fill=uuuuuu] (-1.0642101207295147,1.8031363986642692) circle (2pt);
\draw[color=uuuuuu] (-1.08,1.47) node {$O$};
\draw[color=black] (-1.7,1.97) node {$j$};
\draw[color=black] (-0.48,2.01) node {$i$};
\end{scriptsize}
\end{tikzpicture}

\subsection*{\underline{\textcolor{red}{\textbf{Solution}}}}
\subsection*{\underline{\textcolor{red}{\textbf{c.Milieu d'un segment}}}}
Soit $A(x_{A},y_{A})$ et $A(x_{B},y_{B})$ deux point du plan.\\
Si $I$ est milieu [AB] alors on a:\\
$x_{I}=\frac{x_{A}+x_{B}}{2}$ et $y_{I}=\frac{y_{A}+y_{B}}{2}$
\section*{\underline{\textcolor{red}{\textbf{Reperage et changement de base}}}}
\subsection*{\underline{\textcolor{red}{\textbf{Activité}}}}
Soit ABC un triangle, I milieu de $[BC]$ et M un point défini par : 
$\overrightarrow{AM}=\frac{2}{5}\overrightarrow{AI}$.

La parallèle à (AB) passant par M coupe (BC) en P et la parallèle à (AC) passant par M coupe (BC) en Q. On veut montrer que I est le milieu de [PQ]. Soit le repère $(A\;;\ \overrightarrow{AB}\;,\ \overrightarrow{AC}).$

1) Déterminer les coordonnées de $A\;,\ B\;,\ C\ $ et $\ M.$

2) Vérifier que l'ordonnée de $P$ est $\dfrac{1}{5}.$

3) Soit $x$ l'abscisse de $P$, donner les coordonnées de $\overrightarrow{BP}$ en fonction de $x$. Calculer $x.$

Déterminer les coordonnées de $Q$ puis montrer que $I$ est milieu de $[PQ].$
\subsection*{\underline{\textcolor{red}{\textbf{Correction}}}}

1) Soit le repère $(A\;;\ \overrightarrow{AB}\;,\ \overrightarrow{AC})$ donc, $A\begin{pmatrix} 0 \\ 0\end{pmatrix}\;,\quad B\begin{pmatrix} 1 \\ 0\end{pmatrix}$ et $C\begin{pmatrix} 0 \\ 1\end{pmatrix}$

$\begin{array}{rcl}\overrightarrow{AM}&=&\dfrac{2}{5}\overrightarrow{AI}=\dfrac{2}{5}(\overrightarrow{AB}+\overrightarrow{BI}) \\ \\ &=&\dfrac{2}{5}\left(\overrightarrow{AB}+\dfrac{1}{2}\overrightarrow{BC}\right)\\ \\ &=&\dfrac{2}{5}\left(\overrightarrow{AB}+\dfrac{1}{2}\left(\overrightarrow{BA}+\overrightarrow{AC}\right)\right)\\ \\ &=&\dfrac{2}{5}\left(\overrightarrow{AB}-\dfrac{1}{2}\overrightarrow{AB}+\dfrac{1}{2}\overrightarrow{AC}\right)\\ \\ &=&\dfrac{2}{5}\left(\dfrac{1}{2}\overrightarrow{AB}+\dfrac{1}{2}\overrightarrow{AC}\right)\\ \\ &=&\dfrac{1}{5}\overrightarrow{AB}+\dfrac{1}{5}\overrightarrow{AC}\end{array}$

D'où, $M\begin{pmatrix}\dfrac{1}{5} \\ \\ \dfrac{1}{5}\end{pmatrix}$

2) $(MP)$ parallèle à $(AB)$ qui est l'axe des abscisses donc $M$ et $P$ ont la même ordonnée. D'où : $y_{P}=\dfrac{1}{5}.$
3) $x_{P}=x$ donc, $P\begin{pmatrix} x \\ \dfrac{1}{5}\end{pmatrix}$ et $\overrightarrow{BP}\begin{pmatrix} x-1 \\ \dfrac{1}{5}\end{pmatrix}$

$\overrightarrow{BP}$ est colinéaire à $\overrightarrow{BC}$, donc $B\;,\ P$ et $C$ sont alignés.

Soit $\overrightarrow{BC}\begin{pmatrix} -1 \\ 1\end{pmatrix}$ alors on a : $$\begin{array}{rcl} (x-1)(1)-\left(\dfrac{1}{5}\right)(-1)=0&\Leftrightarrow&x-1+\dfrac{1}{5}=0\\\\&\Leftrightarrow&x=1-\dfrac{1}{5}\\ \\&\Leftrightarrow&x=\dfrac{4}{5}\end{array}$$

D'où  : $P\begin{pmatrix}\dfrac{4}{5} \\ \\ \dfrac{1}{5}\end{pmatrix}$

$(MQ)$ parallèle à $(AC)$ qui est l'axe des ordonnées donc $Q$ a même abscisse que $M.$ D'où : $x_{Q}=\dfrac{1}{5}.$

Soit $Q\begin{pmatrix}\dfrac{1}{5} \\ y_{Q}\end{pmatrix}$ et $\overrightarrow{CQ}\begin{pmatrix}\dfrac{1}{5} \\ y_{Q}-1\end{pmatrix}$

\begin{eqnarray}\overrightarrow{CQ}\text{ colinéaire à }\overrightarrow{BC}&\Leftrightarrow&\left(\dfrac{1}{5}\right)(1)-(y_{Q}-1)(-1)=0\nonumber\\ \\ &\Leftrightarrow&\dfrac{1}{5}+y_{Q}-1=0\nonumber\\ \\ &\Leftrightarrow&y_{Q}=1-\dfrac{1}{5}=\dfrac{4}{5}\nonumber \end{eqnarray}

D'où  : $Q\begin{pmatrix}\dfrac{1}{5} \\ \\ \dfrac{4}{5}\end{pmatrix}$

$I$ est milieu de $[BC]$ alors $I\begin{pmatrix}\dfrac{1}{2} \\ \\ \dfrac{1}{2}\end{pmatrix}$

Calculons les coordonnées du milieu de $[PQ]$

Soit $$\begin{array}{ccc}\dfrac{x_{P}+x_{Q}}{2}&=&\dfrac{\left(\dfrac{4}{5}+\dfrac{1}{5}\right)}{2} \\ \\ &=&\dfrac{1}{2}\ =\ x_{I} \end{array}\quad\text{et}\quad\begin{array}{ccc}\dfrac{y_{P}+y_{Q}}{2}&=&\dfrac{\left(\dfrac{1}{5}+\dfrac{4}{5}\right)}{2} \\ \\ &=&\dfrac{1}{2}\ =\ y_{I} \end{array}$$

3) $x_{P}=x$ donc, $P\begin{pmatrix} x \\ \dfrac{1}{5}\end{pmatrix}$ et $\overrightarrow{BP}\begin{pmatrix} x-1 \\ \dfrac{1}{5}\end{pmatrix}$

$\overrightarrow{BP}$ est colinéaire à $\overrightarrow{BC}$, donc $B\;,\ P$ et $C$ sont alignés.

Soit $\overrightarrow{BC}\begin{pmatrix} -1 \\ 1\end{pmatrix}$ alors on a : $$\begin{array}{rcl} (x-1)(1)-\left(\dfrac{1}{5}\right)(-1)=0&\Leftrightarrow&x-1+\dfrac{1}{5}=0\\\\&\Leftrightarrow&x=1-\dfrac{1}{5}\\ \\&\Leftrightarrow&x=\dfrac{4}{5}\end{array}$$

D'où  : $P\begin{pmatrix}\dfrac{4}{5} \\ \\ \dfrac{1}{5}\end{pmatrix}$

$(MQ)$ parallèle à $(AC)$ qui est l'axe des ordonnées donc $Q$ a même abscisse que $M.$ D'où : $x_{Q}=\dfrac{1}{5}.$

Soit $Q\begin{pmatrix}\dfrac{1}{5} \\ y_{Q}\end{pmatrix}$ et $\overrightarrow{CQ}\begin{pmatrix}\dfrac{1}{5} \\ y_{Q}-1\end{pmatrix}$

\begin{eqnarray}\overrightarrow{CQ}\text{ colinéaire à }\overrightarrow{BC}&\Leftrightarrow&\left(\dfrac{1}{5}\right)(1)-(y_{Q}-1)(-1)=0\nonumber\\ \\ &\Leftrightarrow&\dfrac{1}{5}+y_{Q}-1=0\nonumber\\ \\ &\Leftrightarrow&y_{Q}=1-\dfrac{1}{5}=\dfrac{4}{5}\nonumber \end{eqnarray}
D'où  : $Q\begin{pmatrix}\dfrac{1}{5} \\ \\ \dfrac{4}{5}\end{pmatrix}$
$I$ est milieu de $[BC]$ alors $I\begin{pmatrix}\dfrac{1}{2} \\ \\ \dfrac{1}{2}\end{pmatrix}$

Calculons les coordonnées du milieu de $[PQ]$

Soit $$\begin{array}{ccc}\dfrac{x_{P}+x_{Q}}{2}&=&\dfrac{\left(\dfrac{4}{5}+\dfrac{1}{5}\right)}{2} \\ \\ &=&\dfrac{1}{2}\ =\ x_{I} \end{array}\quad\text{et}\quad\begin{array}{ccc}\dfrac{y_{P}+y_{Q}}{2}&=&\dfrac{\left(\dfrac{1}{5}+\dfrac{4}{5}\right)}{2} \\ \\ &=&\dfrac{1}{2}\ =\ y_{I} \end{array}$$

Donc $I$ est aussi milieu de $[PQ]$
\subsection*{\underline{\textcolor{red}{\textbf{I. Définitions}}}}
Soient deux vecteurs non nuls $\vec{u}$ et $\vec{v}$ et non colinéaires, $A$ un point du plan. Le triplet $(A;\ \vec{u},\ \vec{v})$ est un repère du plan où $A$ est l'origine et $\vec{u}$ et $\vec{v}$ sont les vecteurs de base du plan.

$\centerdot$ $\forall\;$ $M\in\mathcal{P}$ $\exists\;$ $x$ et $y$ tels que $\overrightarrow{AM}=x\vec{u}+y\vec{v}$

$x$ et $y$ sont les coordonnées de $M$. $\ x$ est l'abscisse de $M\;$; $\ x_{M}=x$, $\ y$ est l'ordonnée de $M\;$; $y_{M}=y$

$\centerdot\ \ $ Si $\vec{u}$ est orthogonal à $\vec{v}\ $ ($\vec{u}\perp\vec{v}$) on dira que le repère $(A;\ \vec{u},\ \vec{v})$ est un repère orthogonal.

$\centerdot\ \ $ Si de plus $||\vec{u}||=||\vec{v}||=1$ on dira qu'on a un repère orthonormé.

$\centerdot\ \ (A;\ \vec{u})$ est l'axe des abscisses, $(A;\ \vec{v})$ est l'axe des ordonnées.
\subsection*{\underline{\textcolor{red}{\textbf{II. Équations cartésiennes et paramétriques de droite}}}}
\subsection*{\underline{\textcolor{red}{\textbf{1. Colinéarité de vecteurs}}}}
$\centerdot\ \ $ Deux vecteurs $\vec{u}\begin{pmatrix}
x\\
y
\end{pmatrix}$ et $\vec{v}\begin{pmatrix}
x'\\
y'
\end{pmatrix}$ sont colinéaires si, et seulement si, le déterminant de $\vec{u}$ et $\vec{v}$ noté $$det(\vec{u},\ \vec{v})=\begin{vmatrix}
x & x'\\
	y & y'
	\end{vmatrix}=xy'-x'y=0$$
	$\centerdot\ \ $ Deux vecteurs $\vec{u}$ et $\vec{v}$ sont colinéaires si, et seulement si, il existe $k\in\mathbb{R}$ tel que $\vec{u}=k.\vec{v}$
	
$\vec{u}$ et $\vec{v}$ ont même direction.
\subsection*{\underline{\textcolor{red}{\textbf{2. Vecteurs orthogonaux}}}}
	$\centerdot\ \ $ Deux vecteurs $\vec{u}$ et $\vec{v}$ sont orthogonaux si, et seulement si, ils ont des directions orthogonales. On note $\vec{u}\perp\vec{v}$.
	
	$\centerdot\ \ $ Deux vecteurs $\vec{u}\begin{pmatrix}
	x\\
	y
	\end{pmatrix}$ et $\vec{v}\begin{pmatrix}
	x'\\
	y'
	\end{pmatrix}$ sont orthogonaux si, et seulement si, $$xx'+yy'=0$$
\subsection*{\underline{\textcolor{red}{\textbf{3.1 Définitions}}}}	
Soient $\vec{u}$ un vecteur non nul, $A$ un point du plan. L'ensemble des points $M$ du plan tels que $\overrightarrow{AM}$ colinéaire à $\vec{u}$ est la droite passant par $A$ de direction $\vec{u}$ noté $(D)=(A,\ \vec{u}).$
$\vec{u}$ est un vecteur directeur de $(D)$ et tout vecteur $\vec{v}$ colinéaire à $\vec{u}$ est aussi un vecteur directeur de $(D).$
\subsection*{\underline{\textcolor{red}{\textbf{3.2 Équation cartésienne et réduite d'une droite}}}}
L'équation cartésienne d'une droite $(D)$ est de la forme $$ax+by+c=0$$ avec $(a,\ b)\neq(0,\ 0)\;$; $\ a$ et $b$ ne sont pas nuls en même temps.
	$\centerdot\ \ $ Le vecteur $\vec{u}\begin{pmatrix}
	-b\\
	a
	\end{pmatrix}$ est un vecteur directeur de $(D)$.
	
	$\centerdot\ \ $ Le vecteur $\vec{n}\begin{pmatrix}
	a\\
	b
	\end{pmatrix}$ est appelé un vecteur normal à $(D)$. C'est un vecteur orthogonal à la direction de $(D)$.
	
	$\centerdot\ \ $ Si $a=0$ alors, la droite $(D)$ a pour équation $y=\dfrac{-c}{b}$ et est parallèle à l'axe des abscisses.
	
	$\centerdot\ \ $ Si $b=0$ alors, la droite $(D)$ a pour équation $y=\dfrac{-c}{a}$ et est parallèle à l'axe des ordonnées.
\subsection*{\underline{\textcolor{red}{\textbf{Exercice d'application}}}}
Soit $A\begin{pmatrix} 2 \\ 5\end{pmatrix}\;,\quad B\begin{pmatrix} 3 \\ -4\end{pmatrix}\;,\quad C\begin{pmatrix} 1 \\ 7\end{pmatrix}$

Déterminer les équations cartésiennes de :

1) $(AB)$

2) la droite $(D)$ passant par $C$ et parallèle à $(AB)$

3) la médiatrice de $[AC]$


\subsection*{\underline{\textcolor{red}{\textbf{Résolution}}}}
1) Équation cartésienne de $(AB)$ :

Nous avons : $A\begin{pmatrix} 3-2 \\ -4-5\end{pmatrix}=\begin{pmatrix} 1 \\ -9\end{pmatrix}$

Soit $M\begin{pmatrix} x \\ y\end{pmatrix}\in(AB)\;;\quad \overrightarrow{AM}\begin{pmatrix} x-2 \\ y-5\end{pmatrix}$

$$\begin{array}{rcl} M\begin{pmatrix} x \\ y\end{pmatrix}\in(AB)&\Leftrightarrow&\overrightarrow{AB}\ \text{ colinéaire à }\ \overrightarrow{AM} \\ \\ &\Leftrightarrow&det(\overrightarrow{AM}\;,\ \overrightarrow{AB})=0\\ \\ &\Leftrightarrow&\begin{vmatrix} x-2&1 \\ y-5&-9\end{vmatrix}=(x-2)(-9)-(y-5)=0\\ \\ &\Leftrightarrow&-9x-y+23=0\end{array}$$

Donc, $(AB)\ :\ -9x-y+23=0$

2) Soit $M\begin{pmatrix} x \\ y\end{pmatrix}\in(D)\;;\quad \overrightarrow{CM}\begin{pmatrix} x-1 \\ y-7\end{pmatrix}$

$$\begin{array}{rcl} (D)\ \text{ parallèle à }\ (AB)&\Leftrightarrow&\overrightarrow{AB}\ \text{ colinéaire à }\ \overrightarrow{CM} \\ \\ &\Leftrightarrow&det(\overrightarrow{AB}\;,\ \overrightarrow{CM})=0\\ \\ &\Leftrightarrow&\begin{vmatrix} 1&x-1 \\ -9&y-7\end{vmatrix}=(y-7)-(-9)(x-1)=0\\ \\ &\Leftrightarrow&y+9x-7-9=0 \end{array}$$

D'où, $(D)\ :\ y+9x-16=0$

3) Soit $(\Delta)$ la médiatrice de $[AC]$ donc $\overrightarrow{AC}\begin{pmatrix} -1 \\ 2\end{pmatrix}$ est un vecteur normal à $(\Delta).$

Soit $I$ milieu de $[AC]\;;\quad I\begin{pmatrix}\dfrac{3}{2} \\ \\ 6\end{pmatrix}$ et soit $M\begin{pmatrix} x \\ y\end{pmatrix}\in(\Delta)\;;\quad \overrightarrow{IM}\begin{pmatrix} x-\dfrac{3}{2} \\ \\ y-6\end{pmatrix}$

$$\begin{array}{rcl} (\Delta)\ \text{ perpendiculaire à }\ (AC)&\Leftrightarrow&(-1)\left(x-\dfrac{3}{2}\right)+2(y-6)=0 \\ \\ &\Leftrightarrow&-x+\dfrac{3}{2}+2y-12=0\\ \\ &\Leftrightarrow&-x+2y-\dfrac{21}{2}=0\end{array}$$

Donc, $\Delta\ :\ -x+2y-\dfrac{21}{2}=0$
\subsection*{\underline{\textcolor{red}{\textbf{Remarques}}}}
L'équation réduite d'une droite $(D)$ est de la forme $y=\alpha x+\beta$ où $\alpha$ est le coefficient directeur de la droite $(D)$. Donc si $(D) \: : y=\alpha x+\beta \: \Longrightarrow \: \alpha x-y+\beta=0$ et $\vec{u}\begin{pmatrix} 1\\ \alpha \end{pmatrix}$ est un vecteur directeur de $(D).$

Soit $(D) \: : ax+by+c=0$, si $b\neq 0 \: \Rightarrow \: y=\dfrac{-a}{b}x-\dfrac{c}{b}$ et $\dfrac{-a}{b}$ est le coefficient directeur de la droite $(D).$
\subsection*{\underline{\textcolor{red}{\textbf{3.3 Conditions de parallélisme et d'orthogonalité de droites}}}}
$\centerdot\ \ $ Deux droites $(D) \: : ax+by+c=0$ de vecteur directeur $\vec{u}\begin{pmatrix} -b\\ a \end{pmatrix}$ et $(D') \: : a'x+b'y+c'=0$ de vecteur directeur $\vec{u}'\begin{pmatrix} -b'\\ a' \end{pmatrix}$ sont parallèles si, et seulement si, les vecteurs directeurs $\vec{u}$ et $\vec{u}'$ sont colinéaires ($det(\vec{u},\ \vec{u}')=0$)

$\centerdot\ \ (D)$ et $(D')$ sont orthogonales si, et seulement si, les vecteurs directeurs $\vec{u}$ et $\vec{u}'$ sont orthogonaux ($x_{\vec{u}}x'_{\vec{u}'}+y_{\vec{u}}y'_{\vec{u}'}=0$)

$\centerdot$ Deux droites $(D) \: : y=\alpha x+\beta$ et $(D') \: : y=\alpha' x+\beta'$ sont parallèles si, et seulement si, $$\alpha=\alpha'$$

$\centerdot\ \ (D)$ et $(D')$ sont perpendiculaires si, et seulement si, $$\alpha.\alpha'=-1$$
\subsection*{\underline{\textcolor{red}{\textbf{Exercice d'application}}}}
Soit $(D_{1})\ :\ 2x-3y+5=0$ et $A\begin{pmatrix} 3 \\ 4\end{pmatrix}\;,\quad B\begin{pmatrix} 2 \\ 3\end{pmatrix}$ deux points du plan.

1) $A$ et $B$ appartiennent-ils à $(D_{1})$ ?

2) Déterminer l'équation de la droite $(D_{2})$ passant par $A$ et parallèle à $(D_{1}).$

3) Soit $(D_{3})\ :\ y=3x-1$

$(D_{3})$ et $(D_{1})$ sont-elles parallèles ? perpendiculaires ?
\subsection*{\underline{\textcolor{red}{\textbf{Résolution}}}}
1) $M\begin{pmatrix} x \\ y\end{pmatrix}\in(D_{1})$ si, et seulement si, ses coordonnées vérifient l'équation de $(D_{1})$

$$\begin{array}{rcl} A\begin{pmatrix} 3 \\ 4\end{pmatrix}\ :\quad 2\times 3-3\times 4+5&=&6-12+5\\ \\ &=&-1\ \neq 0\end{array}$$

Donc, $A\notin(D_{1})$

$$\begin{array}{rcl} B\begin{pmatrix} 2 \\ 3\end{pmatrix}\ :\quad 2\times 2-3\times 3+5&=&4-9+5\\ \\ &=&0\end{array}$$

Donc, $B\in(D_{1})$

2) Soit $M\begin{pmatrix} x \\ y\end{pmatrix}\in(D_{2})\;;\quad \overrightarrow{AM}\begin{pmatrix} x-3 \\ y-4\end{pmatrix}$ et soit $\vec{u}\begin{pmatrix} 3 \\ 2\end{pmatrix}$ vecteur directeur de $(D_{1})$ alors

$$\begin{array}{rcl} (D_{2})\ \text{ parallèle à }\ (D_{1})&\Leftrightarrow&\overrightarrow{AM}\ \text{ colinéaire à }\ \vec{u} \\ \\ &\Leftrightarrow&det(\overrightarrow{AM}\;,\ \vec{u})=0\\ \\ &\Leftrightarrow&\begin{vmatrix} x-3&3 \\ y-4&2\end{vmatrix}=(x-3)(2)-(3)(y-4)=0\\ \\ &\Leftrightarrow&2x-3y+6=0\end{array}$$

D'où, $(D_{2})\ :\ 2x-3y+6=0$

3) $(D_{1})\ :\ 2x-3y+5=0$ donc $y=\dfrac{2}{3}x+\dfrac{5}{3}$ est une équation réduite pour $(D_{1}).$

Soient $u_{1}\begin{pmatrix} 1 \\ \\ \dfrac{2}{3}\end{pmatrix}$ et $u_{3}\begin{pmatrix} 1 \\ 3\end{pmatrix}$ vecteurs directeurs respectifs de $(D_{1})$ et $(D_{3}).$

Soient $u_{1}\begin{pmatrix} 1 \\ \\ \dfrac{2}{3}\end{pmatrix}$ et $u_{3}\begin{pmatrix} 1 \\ 3\end{pmatrix}$ vecteurs directeurs respectifs de $(D_{1})$ et $(D_{3}).$

$3\times\dfrac{2}{3}=1\neq -1\;,\ $ alors, $(D_{1})$ et $(D_{3})$ ne sont pas perpendiculaires.

\subsection*{\underline{\textcolor{red}{\textbf{3.4 Équations paramétriques}}}}
Soient $\vec{u}\begin{pmatrix} \alpha\\ \beta \end{pmatrix}$ et $M_{0}\begin{pmatrix} x_{0}\\ y_{0} \end{pmatrix}$, $\ (D)$ est une droite passant par $M_{0}$ de vecteur directeur $\vec{u}\;$; $\ (D)=(M_{0},\ \vec{u}).$

	$M\begin{pmatrix}
	x\\
	y
	\end{pmatrix}\in(D)$ si, et seulement si, $\overrightarrow{M_{0}M}$ est colinéaire à $\vec{u}.$
	Donc, $\exists\;$ $k\in\mathbb{R}$ tel que $\overrightarrow{M_{0}M}=k.\vec{u}.$
	Ainsi, $\left\lbrace\begin{array}{lcl}
	x-x_{0} &=& k\alpha\\
	y-y_{0} &=& k\beta
	\end{array}
	\right. \Longrightarrow  \left\lbrace\begin{array}{lcl}
	x &=& x_{0}+k\alpha\\
	y &=& y_{0}+k\beta
	\end{array}
	\right.$ qui est un système d'équations paramétriques de $(D)$
	$M_{k}\begin{pmatrix}
	x_{0}+\alpha k\\
	y_{0}+\beta k
	\end{pmatrix}$ est le point de paramètre $k$.
\subsection*{\underline{\textcolor{red}{\textbf{Exercice d'application}}}}
Soit $A\begin{pmatrix} 1 \\ 2\end{pmatrix}\;,\quad B\begin{pmatrix} 3 \\ 4\end{pmatrix}$ deux points du plan et $(D_{1})\ :\ 2x-3y+5=0.$

1) Déterminer un système d'équations paramétriques des droites $(AB)$ et $(D_{1}).$

2) Soit $(D_{2})\ :\ \left\lbrace\begin{array}{rcl} x&=&2-t \\ y&=&3+4t\end{array}\right.$

a) $A\begin{pmatrix} 1 \\ 2\end{pmatrix}$ et $D\begin{pmatrix} 0 \\ 11\end{pmatrix}$ appartiennent-ils à $(D_{2})\;\ ?$

b) Déterminer l'équation cartésienne de $(D_{2})$

3) Déterminer les coordonnées des points d'intersection de $(D_{1})$ et $(D_{2})$
\subsection*{\underline{\textcolor{red}{\textbf{Résolution}}}}
Soit $M\begin{pmatrix} x \\ y\end{pmatrix}\in(AB)$ alors $\overrightarrow{AB}$ et $\overrightarrow{AM}$ sont colinéaires.

$$\begin{array}{rcl} \overrightarrow{AB}\ \text{ colinéaire à }\ \overrightarrow{AM}&\Leftrightarrow&\overrightarrow{AM}=t\overrightarrow{AB}\;,\quad t\in\mathbb{R}\\ \\ &\Leftrightarrow&\left\lbrace\begin{array}{rcl} x-1&=&2t \\ y-2&=&2t \end{array}\right.\\ \\ &\Leftrightarrow&\left\lbrace\begin{array}{rcl} x&=&2t+1 \\ y&=&2t+2\end{array}\right.\quad\text{équation paramétrique de }(AB)\end{array}$$

Soit $\vec{u}\begin{pmatrix} 3 \\ 2\end{pmatrix}$ vecteur directeur de $(D_{1})$ et $A\begin{pmatrix} -1 \\ 1\end{pmatrix}$ un point de $(D_{1})$

$M\begin{pmatrix} x \\ y\end{pmatrix}\in(D_{1})$ alors $\overrightarrow{AM}$ et $\vec{u}$ sont colinéaires.

$$\begin{array}{rcl} \overrightarrow{AM}\ \text{ colinéaire à }\ \vec{u}&\Leftrightarrow&\overrightarrow{AM}=t\vec{u}\;,\quad t\in\mathbb{R}\\ \\ &\Leftrightarrow&\left\lbrace\begin{array}{rcl} x+1&=&3t \\ y-1&=&2t \end{array}\right.\\ \\ &\Leftrightarrow&\left\lbrace\begin{array}{rcl} x&=&3t-1 \\ y&=&2t+1\end{array}\right.\quad\text{équation paramétrique de }(D_{1})\end{array}$$

2) a) $\left\lbrace\begin{array}{rcl} 1&=&2-t \\ 2&=&3+4t\end{array}\right.\ \Leftrightarrow\ \left\lbrace\begin{array}{rcl} t&=&1 \\ \\ t&=&-\dfrac{1}{4}\end{array}\right.\quad\text{impossible}$

D'où, $A\notin(D_{2})$

$\left\lbrace\begin{array}{rcl} 0&=&2-t \\ 11&=&3+4t\end{array}\right.\ \Leftrightarrow\ \left\lbrace\begin{array}{rcl} t&=&2 \\ t&=&2\end{array}\right.\quad\text{donc }D\in(D_{2})$

b) $\left\lbrace\begin{array}{rcl} x&=&2-t \\ y&=&3+4t\end{array}\right.\ \Leftrightarrow\ \left\lbrace\begin{array}{rcl} t&=&2-x \\ y&=&3+4(2-x)\ =\ -4x+11\end{array}\right.$

D'où, $y+4x-11=0$ est une équation cartésienne de $(D_{2})$

3) Dans $(D_{1})$, remplaçons $x$ et $y$ par leur valeur dans $(D_{2})$
	$$\begin{array}{rcl} 2(2-t)-3(3+4t)+5=0&\Leftrightarrow&-14t=0\\\\&\Leftrightarrow&t=0\end{array}$$
	
Donc, $\left\lbrace\begin{array}{rcl} x&=&2 \\ y&=&3\end{array}\right.$ ; d'où $(D_{1})\cap(D_{2})=I\begin{pmatrix} 2 \\ 3\end{pmatrix}$
\subsection*{\underline{\textcolor{red}{\textbf{3.5 Équations de cercle}}}}
Soit $\mathcal{C}$ de $I\begin{pmatrix}
	\alpha\\
	\beta
	\end{pmatrix}$ et de rayon $R$ noté $\mathcal{C}(I,\ R).$
	$M\begin{pmatrix}
	x\\
	y
	\end{pmatrix}\in\mathcal{C}(I,\ R) \: \Rightarrow IM=R.$
	
Donc, $IM^{2}=R^{2} \: \Longrightarrow (x_{M}-x_{I})^{2}+(y_{M}-y_{I})^{2}=R^{2}.$
	D'où : $\ (x_{M}-\alpha)^{2}+(y_{M}-\beta)^{2}=R^{2}$ qui est l'équation réduite du cercle $\mathcal{C}\left(I\begin{pmatrix}
	\alpha\\
	\beta
	\end{pmatrix}, R\right).$
	$$x^{2}+y^{2}-2\alpha x-2\beta y+\alpha^{2}+\beta^{2}-R^{2}=0$$ est l'équation cartésienne du cercle $(\mathcal{C}).$
\subsection*{\underline{\textcolor{red}{\textbf{Exemple}}}}
Les équations suivantes sont-elles des équations de cercle. Si oui, donner le centre et le rayon.

a) $x^{2}+y^{2}-6x+8y-5=0$

b) $x^{2}+y^{2}+2x+4y-1=0$

c) $x^{2}+y^{2}+3x+4y+20=0$
\subsection*{\underline{\textcolor{red}{\textbf{Résolution}}}}
a) $$\begin{array}{rcl} x^{2}+y^{2}-6x+8y-5=0&\Leftrightarrow&(x-3)^{2}-9+(y+4)^{2}-16-5=0\\ \\ &\Leftrightarrow&(x-3)^{2}+(y+4)^{2}=30\end{array}$$

Donc, cette équation définie bien un cercle $\mathcal{C}\left(I\begin{pmatrix} 3 \\ -4\end{pmatrix}\;;\ \sqrt{30}\right)$

b) $$\begin{array}{rcl} x^{2}+y^{2}+2x+4y-1=0&\Leftrightarrow&(x+1)^{2}-1+(y+2)^{2}-4-1=0\\ \\ &\Leftrightarrow&(x+1)^{2}+(y+2)^{2}=6\end{array}$$

On obtient bien un cercle $\mathcal{C}\left(I\begin{pmatrix} -1 \\ -2\end{pmatrix}\;;\ \sqrt{6}\right)$

c) $$\begin{array}{rcl} x^{2}+y^{2}+3x+4y+20=0&\Leftrightarrow&\left(x+\dfrac{3}{2}\right)^{2}-\dfrac{9}{4}+(y+2)^{2}-4+20=0\\ \\ &\Leftrightarrow&\left(x+\dfrac{3}{2}\right)^{2}+(y+2)^{2}=-\dfrac{55}{4}\end{array}$$

$-\dfrac{55}{4}<0$ donc cette équation ne définie pas un cercle.
\subsection*{\underline{\textcolor{red}{\textbf{II. Formule de changement de repère}}}}
Soient $(O;\  \vec{i},\  \vec{j})$ un repère du plan, $S$ un point de coordonnées $(a,\ b)$ , dans $(O;\  \vec{i},\  \vec{j}).$
	Considérons un nouveau repère $(S;\  \vec{i},\  \vec{j}).$
	Donc, $M\begin{pmatrix}
	x\\
	y
	\end{pmatrix}$ dans $(O;\  \vec{i},\  \vec{j})$ et $M\begin{pmatrix}
	X\\
	Y
	\end{pmatrix}$ dans $(S;\  \vec{i},\  \vec{j})$
	
		Nous obtenons $\left\lbrace\begin{array}{lcl}
		x &=&X+a\\
		y &=&Y+b
		\end{array}
		\right.$ qui constitue la formule de changement de repère.
\subsection*{\underline{\textcolor{red}{\textbf{Exemple}}}}
Soit $(O\;;\ \vec{i}\;,\ \vec{j})$ un repère du plan, $A\begin{pmatrix} 2 \\ 1\end{pmatrix}\;,\ B\begin{pmatrix} 3 \\ 4\end{pmatrix}$ et $\Omega(3\;;\ 2)$ dans $(O\;;\ \vec{i}\;,\ \vec{j}).$

Trouver les coordonnées de $A$ et $B$ dans $(\Omega\;;\ \vec{i}\;,\ \vec{j}).$
\subsection*{\underline{\textcolor{red}{\textbf{Résolution}}}}
D'après la formule de changement de repère on a : $$\left\lbrace\begin{array}{rcl} x&=&X+3 \\ y&=&Y+2\end{array}\right.\ \Rightarrow\ \left\lbrace\begin{array}{rcl} X&=&x-3 \\ Y&=&y-2\end{array}\right.$$

Les coordonnées de $A$ dans $(\Omega\;;\ \vec{i}\;,\ \vec{j})$ sont donc données par : $$A\begin{pmatrix} X=2-3 \\ Y=1-2\end{pmatrix}=\begin{pmatrix} -1 \\ -1\end{pmatrix}$$

$$\begin{array}{rcl}\overrightarrow{OB}=\overrightarrow{O\Omega}+\overrightarrow{\Omega B}&\Rightarrow&\overrightarrow{\Omega B}=\overrightarrow{OB}-\overrightarrow{O\Omega}\\ \\ &\Rightarrow&\overrightarrow{\Omega B}=3\vec{i}+4\vec{j}-3\vec{i}-2\vec{j}\\ \\ &\Rightarrow&\overrightarrow{\Omega B}=2\vec{j}\end{array}$$

D'où : $B\begin{pmatrix} 0 \\ 2\end{pmatrix}$ dans $(\Omega\;;\ \vec{i}\;,\ \vec{j}).$ 
\end{document}