\documentclass[12pt]{article}
\usepackage{stmaryrd}
\usepackage{graphicx}
\usepackage[utf8]{inputenc}

\usepackage[french]{babel}
\usepackage[T1]{fontenc}
\usepackage{hyperref}
\usepackage{verbatim}

\usepackage{color, soul}

\usepackage{pgfplots}
\pgfplotsset{compat=1.15}
\usepackage{mathrsfs}

\usepackage{amsmath}
\usepackage{amsfonts}
\usepackage{amssymb}
\usepackage{tkz-tab}
\author{Destinés à la 2ndS\\Au Lycée de Dindéferlo}
\title{\textbf{Equations-Inequation du 2nd Degré}}
\date{\today}
\usepackage{tikz}
\usetikzlibrary{arrows, shapes.geometric, fit}

% Commande pour la couleur d'accentuation
\newcommand{\myul}[2][black]{\setulcolor{#1}\ul{#2}\setulcolor{black}}
\newcommand\tab[1][1cm]{\hspace*{#1}}

\begin{document}
\maketitle
\newpage
\section*{\underline{\textbf{\textcolor{red}{I.Trinome Du second degré}}}}
\subsection*{\underline{\textbf{\textcolor{red}{1.Forme canonique}}}}
\subsection*{\underline{\textbf{\textcolor{red}{2.Factorisation de la forme canonique}}}}
\section*{\underline{\textbf{\textcolor{red}{II.Equation second degré}}}}
\subsection*{\underline{\textbf{\textcolor{red}{1.Résolution de l'équation $ax^{2}+bx+c=0$}}}}
\subsection*{\underline{\textbf{\textcolor{red}{Théorème1}}}}
$\ast$\underline{\textbf{\textcolor{red}{Exemple}}}\\
\subsection*{\underline{\textbf{\textcolor{red}{Théorème2}}}}
$\ast$\underline{\textbf{\textcolor{red}{Exemple}}}\\
\subsection*{\underline{\textbf{\textcolor{red}{Théorème3}}}}
$\ast$\underline{\textbf{\textcolor{red}{Exemple}}}\\
\subsection*{\underline{\textbf{\textcolor{red}{2.Somme et Produit des Racines}}}}
Soit $x_{1}$;$x_{2}$ les racines de l'équation $ax^{2}+bx+c=0$\\
$\ast$\underline{\textbf{\textcolor{red}{Somme}}}\\
$\ast$\underline{\textbf{\textcolor{red}{Exemple}}}\\
$\ast$\underline{\textbf{\textcolor{red}{Produit}}}\\
$\ast$\underline{\textbf{\textcolor{red}{Exemple}}}\\
\subsection*{\underline{\textbf{\textcolor{red}{3.Equation se ramenant à une équation du second degré}}}}
$\ast$\underline{\textbf{\textcolor{red}{Equation du type $ax^{4}+bx^{2}+c=0$}}}\\
$\ast$\underline{\textbf{\textcolor{red}{Exemple}}}\\
$\ast$\underline{\textbf{\textcolor{red}{Equation du type $ax^{2}+b|x|+c=0$}}}\\
$\ast$\underline{\textbf{\textcolor{red}{Exemple}}}\\
\section*{\underline{\textbf{\textcolor{red}{III.Inequation du second degré}}}}
\subsection*{\underline{\textbf{\textcolor{red}{Théorème3}}}}
\end{document}