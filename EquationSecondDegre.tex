\documentclass[12pt]{article}
\usepackage{stmaryrd}
\usepackage{graphicx}
\usepackage[utf8]{inputenc}

\usepackage[french]{babel}
\usepackage[T1]{fontenc}
\usepackage{hyperref}
\usepackage{verbatim}

\usepackage{color, soul}

\usepackage{pgfplots}
\pgfplotsset{compat=1.15}
\usepackage{mathrsfs}

\usepackage{amsmath}
\usepackage{amsfonts}
\usepackage{amssymb}
\usepackage{tkz-tab}
\author{Destinés à la 2ndS\\Au Lycée de Dindéferlo}
\title{\textbf{Equations-Inequation du 2nd Degré}}
\date{\today}
\usepackage{tikz}
\usetikzlibrary{arrows, shapes.geometric, fit}

% Commande pour la couleur d'accentuation
\newcommand{\myul}[2][black]{\setulcolor{#1}\ul{#2}\setulcolor{black}}
\newcommand\tab[1][1cm]{\hspace*{#1}}

\begin{document}
\maketitle
\newpage
\section*{\underline{\textbf{\textcolor{red}{I.Trinôme du second degré}}}}
\subsection*{\underline{\textbf{red}{1. Définition}}}
On appelle trinôme du second degré, toute expression de la forme

$ ax^{2}+bx+c $; $a$, $b$ et $c$ $\in \mathbb{R}$; $\neq 0 $
\subsection*{\underline{\textbf{\textcolor{red}{2.Forme canonique}}}}
Soit $f(x)= ax^{2}+bx+c $ avec $\neq 0$. On a:
\[f(x)=a\left[ x^{2}+\frac{b}{a}x+\frac{c}{a}\right] \] or 
\[x^{2}+\frac{b}{a}x=\left( x+\frac{b}{2a}\right) ^{2}-\frac{b^{2}}{4a^{2}}\]
Donc, \[f(x)=a\left[\left( x+\frac{b}{2a}\right) ^{2}-\frac{b^{2}}{4a^{2}}+\frac{c}{a}\right] \] 
\[f(x)=a\left[\left( x+\frac{b}{2a}\right) ^{2}-\left( \frac{b^{2}-4ac}{4a^{2}}\right) \right] \]
Ainsi, la forme canonique de $f(x)$ est donnée par \[f(x)=a\left[\left( x+\frac{b}{2a}\right) ^{2}-\left( \frac{b^{2}-4ac}{4a^{2}}\right) \right] \]

En posant $\Delta = b^{2}-4ac $ on a:
\[ f(x)=a \left[ \left(x+\frac{b}{2a}\right)^{2}-\frac{\Delta}{4a^{2}} \right] \]
\subsection*{\underline{\textbf{\textcolor{red}{Exercice d'application}}}}
Déterminer les formes canoniques de 
\[f(x)=3x^{2}-4x+5\]
\[f(x)=-2x^{2}-7x-9\]
\[f(x)=x^{2}-6x+7\]
\[f(x)=-x^{2}+6x-5\]
\subsection*{\underline{\textbf{\textcolor{red}{3.Factorisation de la forme canonique}}}}
\begin{itemize}
\item \textbf{Si $\Delta > 0 $}

\[f(x)=a\left[\left( x+\frac{b}{2a}\right)^{2}-\frac{\Delta}{4a^{2}}\right] \]
\[f(x)=a\left[\left( x+\frac{b}{2a}\right)^{2}-\left( \frac{\sqrt{\Delta}}{2a}\right)^{2} \right]\]
\[f(x)=a\left( x+\frac{b}{2a}-\frac{\sqrt{\Delta}}{2a}\right)\left( x+\frac{b}{2a}+\frac{\sqrt{\Delta}}{2a}\right)\]
\[
\textcolor{red}{
f(x)=a\left(x+\frac{b-\sqrt{\Delta}}{2a} \right)\left(x+\frac{b+\sqrt{\Delta}}{2a} \right)
}
\]

\item \textbf{Si $\Delta < 0 $}, le trinôme n'est pas factorisable

\item \textbf{Si $\Delta = 0 $} alors \[f(x)=a\left( x+\frac{b}{2a}\right)^{2}\] 
\[\textcolor{red}{
f(x)=a\left( x+\frac{b}{2a}\right) \left( x+\frac{b}{2a}\right)
}
\]
\end{itemize}
\section*{\underline{\textbf{\textcolor{red}{II.Equation second degré}}}}
\subsection*{\underline{\textbf{\textcolor{red}{1.Résolution de l'équation $ax^{2}+bx+c=0$}}}}
\subsection*{\underline{\textbf{\textcolor{red}{Théorème1}}}}
Pour résoudre l'équation $ax^{2}+bx+c=0$, on cherche $\Delta$ avec $\Delta=b^{2}-4ac$.
Trois cas se présentent\\
\begin{itemize}
\item Si $\Delta >0$ alors
$ax^{2}+bx+c=0 \Leftrightarrow a\left(x+\frac{b-\sqrt{\Delta}}{2a} \right)\left(x+\frac{b+\sqrt{\Delta}}{2a} \right)=0$
Comme $\neq 0$ alors 

$ \left(x-\frac{b-\sqrt{\Delta}}{2a} \right)=0$ ou 
$\left(x+\frac{b+\sqrt{\Delta}}{2a} \right)=0$

$x=\frac{-b+\sqrt{\Delta}}{2a}$ ou $x=\frac{-b-\sqrt{\Delta}}{2a}$

On note $x_{1}=\frac{-b-\sqrt{\Delta}}{2a}$, $x_{2}=\frac{-b+\sqrt{\Delta}}{2a}$

Ainsi, $S=\left\lbrace \frac{-b-\sqrt{\Delta}}{2a}, \frac{-b+\sqrt{\Delta}}{2a} \right\rbrace $
\item Si $\Delta < 0$ alors Pas de factorisatrion de $S=\emptyset$
\item Si $\Delta = 0$ alors 

$\left( x+\frac{b}{2a}\right)^{2}=0$

$\Leftrightarrow \left( x+\frac{b}{2a}\right) \left( x+\frac{b}{2a}\right)=0$ 

$\Leftrightarrow x=-\frac{b}{2a}$ ou $x=-\frac{b}{2a}$.
l'équation admet une solution double notée $x_{0}=-\frac{b}{2a}$

$ S = \left\lbrace -\frac{b}{2a} \right\rbrace  $
\end{itemize}
$\ast$\underline{\textbf{\textcolor{red}{Exemple}}}\\
\subsection*{\underline{\textbf{\textcolor{red}{Théorème2}}}}
Soit l'équation $ax^{2}+bx+c=0$ et $b=2b'$ donc $\Delta'=b'^{2}-ac$
$\Delta'$ est appelé discriminant réduit.
\begin{itemize}
    \item Si $\Delta' >0$ alors l'équation a 2 racines distinctes $x_{1}$ et $x_{2}$ tels que:
            \[x_{1}=\frac{-b'-\sqrt{\Delta'}}{a}\] et \[x_{2}=\frac{-b'+\sqrt{\Delta'}}{a}\]
            Ainsi $S=\left\lbrace \frac{-b'-\sqrt{\Delta'}}{a}, \frac{-b'+\sqrt{\Delta'}}{a} \right\rbrace$
     \item Si $\Delta' =0$ alors l'équation a une racine double $x_{0}$ tels que:
            \[x_{0}=\frac{-b'}{a}\]
            Ainsi $S=\left\lbrace \frac{-b'}{a} \right\rbrace$
      \item Si $\Delta' <0$ alors l'équation n'admet pas de solution
            Ainsi $S=\emptyset$
        \end{itemize}
$\ast$\underline{\textbf{\textcolor{red}{Exemple}}}\\
\subsection*{\underline{\textbf{\textcolor{red}{Théorème3}}}}
Soit l'équation $ax^{2}+bx+c=0$ et $a+b+c=0$ alors l'équation admet deux solutions $x_{1}$ et $x_{2}$ tel que $x_{1}=1$ et $x_{2}=\frac{c}{a}$

Ainsi, $S=\left\lbrace 1; \frac{c}{a} \right\rbrace $

$\ast$\underline{\textbf{\textcolor{red}{Exemple}}}\\
Redoudre dans $\mathbb{R}$

$x^{2}-5x+4=0$ ; $2x^{2}+4x+2=0$ ; $x^{2}+x+5=0$ ; $4x^{2}+4x+1=0$
\subsection*{\underline{\textbf{\textcolor{red}{2.Somme et Produit des Racines}}}}
Soit $x_{1}$;$x_{2}$ les racines de l'équation $ax^{2}+bx+c=0$\\
$\ast$\underline{\textbf{\textcolor{red}{Somme}}}\\
En notant la somme des racines $S$ c'est-à-dire $S=x_{1}+x_{2}$

On a: $S=\frac{-b-\sqrt{\Delta}}{2a}+\frac{-b+\sqrt{\Delta}}{2a}$

	\textbf{\textcolor{red}{$S=\frac{-b}{a}$}}
	
$\ast$\underline{\textbf{\textcolor{red}{Exemple}}}\\
$\ast$\underline{\textbf{\textcolor{red}{Produit}}}\\
En notant le produit des racines $P$ c'est-à-dire $P=x_{1}\times x_{2}$

On a: $P=\frac{-b-\sqrt{\Delta}}{2a}\times \frac{-b+\sqrt{\Delta}}{2a}$

	\textbf{\textcolor{red}{$P=\frac{c}{a}$}}
	
$\ast$\underline{\textbf{\textcolor{red}{Exemple}}}\\
Soit le trinôme du $2^{nd}$ degré = $p(x)=x^{2}-5x+2$

Sans calculer $x_{1}$ et $x_{2}$, calculer $x_{1}+x_{2}$ ; $x_{1}\times x_{2}$ ; 
$x_{1}^{2}+x_{2}^{2}$ ; $\frac{1}{x_{1}}+\frac{1}{x_{2}}$ ; $x_{1}^{3}+x_{2}^{3}$
\subsection*{\underline{\textbf{\textcolor{red}{3.Equation se ramenant à une équation du second degré}}}}
$\ast$\underline{\textbf{\textcolor{red}{Equation du type $ax^{4}+bx^{2}+c=0$}}}\\
Pour résoudre ce type d'équation, on pose $t=x^{2}$ et $t^{2}=x^{4}$

Donc ça revient a résoudre : $at^{2}+bt+c=0$

$\ast$\underline{\textbf{\textcolor{red}{Exemple}}}\\
Résolver $x^{4}-5x^{2}+4=0$

$\ast$\underline{\textbf{\textcolor{red}{Solution}}}\\
En posant $t=x^{2}$ et $t^{2}=x^{4}$ on a:

$t^{2}-5x+4=0$ en remarquand que $a+b+c=1-5+4=0$ donc ...

$\ast$\underline{\textbf{\textcolor{red}{Equation du type $ax^{2}+b|x|+c=0$}}}\\
$ax^{2}+b|x|+c=0$ or $x^{2}=|x|^{2}$

Donc $ax^{2}+b|x|+c=0 \Leftrightarrow a|x|^{2}+b|x|+c=0$

En posant $t=|x|$ et $t^{2}=|x|^{2}$

On aura $at^{2}+bt+c=0$

$\ast$\underline{\textbf{\textcolor{red}{Exemple}}}\\
Résolvez dans $\mathbb{R}$\\
$x^{2}-3|x|+2=0$
$\ast$\underline{\textbf{\textcolor{red}{Remarque}}}\\
Pour l'équation du type $ax+b\sqrt{x}+c$

On pose $t=\sqrt{x}$ et $t^{2}=x$
\section*{\underline{\textbf{\textcolor{red}{III.Inequation du second degré}}}}
\subsection*{\underline{\textbf{\textcolor{red}{Théorème:}}}}
Soit le trinôme du second degré $p(x)=ax^{2}+bx+c$ et $\Delta=b^{2}-4ac$
\begin{itemize}
\item[*] Si $\Delta >0$ alors on aura deux solutions distinctes $x_{1}$ et $x_{2}$ avec $x_{1} < x_{2}$

\textbf{Dessin}

\item[*] Si $\Delta =0$ on a une solution double $x_{0}$

\textbf{Dessin}

\item[*] Si $\Delta <0$ pas de solutions
\end{itemize}
\section*{\underline{\textbf{\textcolor{red}{V.Equation paramétrique}}}}
\subsection*{\underline{\textbf{\textcolor{red}{Propriété:}}}}
Soit l'équation du second degré : 

$ax^{2}+bx+c=0$ , si $\Delta \geq 0$ alors les solutions $x_{1}$ et $x_{2}$ de singe $S$ et de $P$.
\begin{itemize}
\item[*] Si $P < 0$ alors les solutions $x_{1}$ et $x_{2}$ sont de singes \textcolor{red}{contraires}
\item[*] Si $P>0$ et $S>0$ alors les solutions $x_{1}$ et $x_{2}$ sont  \textcolor{red}{positives}
\item[*] Si $P>0$ et $S<0$ alors les solutions $x_{1}$ et $x_{2}$ sont  \textcolor{red}{négatives}
\item[*] Si $S=0$ alors les solutions $x_{1}$ et $x_{2}$ sont  \textcolor{red}{opposées}
\item[*] Si $P=0$  alors $x_{1}=0$ et $x_{2}=S$
\item[*] Si $P=0$ et $S=0$ alors  $x_{1}=x_{2}=0$
\end{itemize}
\subsection*{\underline{\textbf{\textcolor{red}{Exemple:}}}}
Dans chacun des cas suivants étudier suivant les valeurs du paramètre réel m, l'existance et le signe des racines des équations:

a)
$6x^{2}+5(m-1)x+(m-1)^{2}=0$

b)$(m-2)x^{2}-2(m-5)x+m+2=0$
\subsection*{\underline{\textbf{\textcolor{red}{Solution:}}}}
a)
$6x^{2}+5(m-1)x+(m-1)^{2}=0$

$a=6$ , $b=5(m-1)$ , $c=(m-1)^{2}$

Comme $a=6\neq 0$ alors on a une équation du second degré.

\textcolor{red}{Calculons $\Delta$:}

$\Delta=$

\textcolor{red}{Etudions le signe de $\Delta$:}

\textcolor{red}{Calculons $P$ :}

\textcolor{red}{Calculons $P$ :}

\textcolor{red}{Tableau récapitulatif}

b)$(m-2)x^{2}-2(m-5)x+m+2=0$

$a=(m-2)$ , $b=-2(m-5)$ , $c=m+2$

pour $a=0$, $(m-2) \Leftrightarrow m=2$.

L'équation devient $6x+4=0$\\
.
.
.

pour $a\neq 0$

\textcolor{red}{Calculons $\Delta$:}

$\Delta_{m}=(m-5)^{2}-(m-2)(m+2)$

$\Delta_{m}=m^{2}-10m+25-(m^{2}-4)$

$\Delta_{m}=m^{2}-10m+25-m^{2}+4$

$\Delta_{m}=-10m+29$

\textcolor{red}{Etudions le signe de $\Delta$:}

Posons $\Delta_{m}=0 \Rightarrow -10m+29=0$

$-10m+29=0 \Rightarrow m=\frac{29}{10}$

\definecolor{cqcqcq}{rgb}{0.7529411764705882,0.7529411764705882,0.7529411764705882}
\begin{tikzpicture}[line cap=round,line join=round,>=triangle 45,x=1cm,y=1cm]
%\draw [color=cqcqcq,, xstep=1cm,ystep=1cm] (-7,-10) grid (-22,17);
\clip(-22,3) rectangle (12,10);
\draw [line width=2pt] (-23,8)-- (-7,8); %première ligne A(-22,8)---B(-7,8)
\draw [line width=2pt] (-22,6)-- (-7,6); %deuxième ligne
\draw [line width=2pt] (-22,4)-- (-7,4); %troisième ligne
\draw [line width=2pt] (-22,4)-- (-22,8); %première colonne (-22,4)<----(-22,8);
\draw [line width=2pt] (-18,8)-- (-18,4); %deuxième colone  (-18,8)--->(-18,4);
\draw [line width=2pt] (-7,8)-- (-7,4); %troisième colonne (-7,8)-->(-7,4);
\draw (-22,5.5) node[anchor=north west] {$\Delta_{m}=-10m+29$};
\draw (-21,7) node[anchor=north west] {$m$};
\draw (-18,7) node[anchor=north west] {$-\infty$};
\draw (-8,7) node[anchor=north west] {$+\infty$};
\draw (-15.8,5.3) node[anchor=north west] {$+$};
\draw (-10.5,5.3) node[anchor=north west] {$-$};
\draw [line width=2pt] (-13,6)-- (-13,4); %(-13,6)-- (-13,4);
\draw (-13.2,7) node[anchor=north west] {$\frac{29}{10}$};
\end{tikzpicture}

\textcolor{red}{Calculons $P$ :}

$S=-\frac{-2(m-5)}{m-2}=\frac{2(m-5)}{m-2}$

\textcolor{red}{Calculons $P$ :}

$P=\frac{m+2}{m-2}$

\textcolor{red}{Tableau récapitulatif}

\definecolor{cqcqcq}{rgb}{0.7529411764705882,0.7529411764705882,0.7529411764705882}
\begin{tikzpicture}[line cap=round,line join=round,>=triangle 45,x=1cm,y=1cm]
%\draw [color=cqcqcq,, xstep=1cm,ystep=1cm] (-7,-10) grid (-22,17);
\clip(-22,0) rectangle (12,10);
\draw [line width=2pt] (-23,8)-- (-7,8); %première ligne A(-22,8)---B(-7,8)
\draw [line width=2pt] (-22,6)-- (-7,6); %deuxième ligne
\draw [line width=2pt] (-22,4)-- (-7,4); %troisième ligne
\draw [line width=2pt] (-22,2)-- (-7,2); %quatième ligne
\draw [line width=2pt] (-22,0)-- (-7,0); %cinquième ligne
\draw [line width=2pt] (-22,0)-- (-22,8); %première colonne (-22,4)<----(-22,8);
\draw [line width=2pt] (-18,8)-- (-18,0); %deuxième colone  (-18,8)--->(-18,4);
\draw [line width=2pt] (-15,6)-- (-15,0); %troisième colonne (-13,6)--> (-13,4);

\draw [line width=2pt] (-13,6)-- (-13,0); %quatrième colonne (-13,6)-- (-13,4);
\draw [line width=2pt] (-12.9,6)-- (-12.9,0); %quatrième colonne (-13,6)-- (-13,4);

\draw [line width=2pt] (-11,6)-- (-11,0); %cinquième colonne (-13,6)-- (-13,4);
\draw [line width=2pt] (-9,6)-- (-9,0); %sixième colonne (-13,6)-- (-13,4);
\draw [line width=2pt] (-7,8)-- (-7,0); %Sepième colonne (-7,8)-->(-7,4);
\draw (-22,5.5) node[anchor=north west] {$\Delta_{m}=-10m+29$};
\draw (-22,3) node[anchor=north west] {$S=\frac{2(m-5)}{m-2}$}; 
\draw (-22,1.5) node[anchor=north west] {$P=\frac{m+2}{m-2}$};
\draw (-21,7) node[anchor=north west] {$m$};
\draw (-18,7) node[anchor=north west] {$-\infty$};
\draw (-8,7) node[anchor=north west] {$+\infty$};
%Signe de Delta
\draw (-16.8,5.3) node[anchor=north west] {$+$};
\draw (-14.5,5.3) node[anchor=north west] {$+$};
\draw (-12.5,5.3) node[anchor=north west] {$+$};
\draw (-10.5,5.3) node[anchor=north west] {$-$};
\draw (-8.5,5.3) node[anchor=north west] {$-$};
%Signe de S
\draw (-16.8,3.3) node[anchor=north west] {$-$};
\draw (-14.5,3.3) node[anchor=north west] {$-$};
\draw (-12.5,3.3) node[anchor=north west] {$-$};
\draw (-10.5,3.3) node[anchor=north west] {$-$};
\draw (-8.5,3.3) node[anchor=north west] {$+$};
%Signe de P
\draw (-16.8,1.3) node[anchor=north west] {$-$};
\draw (-14.5,1.3) node[anchor=north west] {$+$};
\draw (-12.5,1.3) node[anchor=north west] {$+$};
\draw (-10.5,1.3) node[anchor=north west] {$+$};
\draw (-8.5,1.3) node[anchor=north west] {$+$};
\draw (-13.2,7) node[anchor=north west] {$2$};
\draw (-11.5,7) node[anchor=north west] {$\frac{29}{10}$};
\draw (-15.2,7) node[anchor=north west] {$-2$};
\draw (-9.5,7) node[anchor=north west] {$5$};
%++++++++++++++++++++++++++++++++++++++
\draw (-15.3,1.5) node[anchor=north west] {$O$};
\draw (-11.3,5.3) node[anchor=north west] {$O$};
\draw (-9.3,3.5) node[anchor=north west] {$O$};
\end{tikzpicture}

\end{document}