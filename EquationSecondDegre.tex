\documentclass[12pt]{article}
\usepackage{stmaryrd}
\usepackage{graphicx}
\usepackage[utf8]{inputenc}

\usepackage[french]{babel}
\usepackage[T1]{fontenc}
\usepackage{hyperref}
\usepackage{verbatim}

\usepackage{color, soul}

\usepackage{pgfplots}
\pgfplotsset{compat=1.15}
\usepackage{mathrsfs}

\usepackage{amsmath}
\usepackage{amsfonts}
\usepackage{amssymb}
\usepackage{tkz-tab}
\author{Destinés à la 2ndS\\Au Lycée de Dindéferlo}
\title{\textbf{Equations-Inequation du 2nd Degré}}
\date{\today}
\usepackage{tikz}
\usetikzlibrary{arrows, shapes.geometric, fit}

% Commande pour la couleur d'accentuation
\newcommand{\myul}[2][black]{\setulcolor{#1}\ul{#2}\setulcolor{black}}
\newcommand\tab[1][1cm]{\hspace*{#1}}

\begin{document}
\maketitle
\newpage
\section*{\underline{\textbf{\textcolor{red}{I.Trinôme du second degré}}}}
\subsection*{\underline{\textbf{red}{1. Définition}}}
On appelle trinôme du second degré, toute expression de la forme

$ ax^{2}+bx+c $; $a$, $b$ et $c$ $\in \mathbb{R}$; $\neq 0 $
\subsection*{\underline{\textbf{\textcolor{red}{2.Forme canonique}}}}
Soit $f(x)= ax^{2}+bx+c $ avec $\neq 0$. On a:
\[f(x)=a\left[ x^{2}+\frac{b}{a}x+\frac{c}{a}\right] \] or 
\[x^{2}+\frac{b}{a}x=\left( x+\frac{b}{2a}\right) ^{2}-\frac{b^{2}}{4a^{2}}\]
Donc, \[f(x)=a\left[\left( x+\frac{b}{2a}\right) ^{2}-\frac{b^{2}}{4a^{2}}+\frac{c}{a}\right] \] 
\[f(x)=a\left[\left( x+\frac{b}{2a}\right) ^{2}-\left( \frac{b^{2}-4ac}{4a^{2}}\right) \right] \]
Ainsi, la forme canonique de $f(x)$ est donnée par \[f(x)=a\left[\left( x+\frac{b}{2a}\right) ^{2}-\left( \frac{b^{2}-4ac}{4a^{2}}\right) \right] \]

En posant $\Delta = b^{2}-4ac $ on a:
\[ f(x)=a \left[ \left(x+\frac{b}{2a}\right)^{2}-\frac{\Delta}{4a^{2}} \right] \]
\subsection*{\underline{\textbf{\textcolor{red}{Exercice d'application}}}}
Déterminer les formes canoniques de 
\[f(x)=3x^{2}-4x+5\]
\[f(x)=-2x^{2}-7x-9\]
\[f(x)=x^{2}-6x+7\]
\[f(x)=-x^{2}+6x-5\]
\subsection*{\underline{\textbf{\textcolor{red}{3.Factorisation de la forme canonique}}}}
\begin{itemize}
\item \textbf{Si $\Delta > 0 $}

\[f(x)=a\left[\left( x+\frac{b}{2a}\right)^{2}-\frac{\Delta}{4a^{2}}\right] \]
\[f(x)=a\left[\left( x+\frac{b}{2a}\right)^{2}-\left( \frac{\sqrt{\Delta}}{2a}\right)^{2} \right]\]
\[f(x)=a\left( x+\frac{b}{2a}-\frac{\sqrt{\Delta}}{2a}\right)\left( x+\frac{b}{2a}+\frac{\sqrt{\Delta}}{2a}\right)\]
\[
\textcolor{red}{
f(x)=a\left(x+\frac{b-\sqrt{\Delta}}{2a} \right)\left(x+\frac{b+\sqrt{\Delta}}{2a} \right)
}
\]

\item \textbf{Si $\Delta < 0 $}, le trinôme n'est pas factorisable

\item \textbf{Si $\Delta = 0 $} alors \[f(x)=a\left( x+\frac{b}{2a}\right)^{2}\] 
\[\textcolor{red}{
f(x)=a\left( x+\frac{b}{2a}\right) \left( x+\frac{b}{2a}\right)
}
\]
\end{itemize}
\section*{\underline{\textbf{\textcolor{red}{II.Equation second degré}}}}
\subsection*{\underline{\textbf{\textcolor{red}{1.Résolution de l'équation $ax^{2}+bx+c=0$}}}}
\subsection*{\underline{\textbf{\textcolor{red}{Théorème1}}}}
Pour résoudre l'équation $ax^{2}+bx+c=0$, on cherche $\Delta$ avec $\Delta=b^{2}-4ac$.
Trois cas se présentent\\
\begin{itemize}
\item Si $\Delta >0$ alors
$ax^{2}+bx+c=0 \Leftrightarrow a\left(x+\frac{b-\sqrt{\Delta}}{2a} \right)\left(x+\frac{b+\sqrt{\Delta}}{2a} \right)=0$
Comme $\neq 0$ alors 

$ \left(x-\frac{b-\sqrt{\Delta}}{2a} \right)=0$ ou 
$\left(x+\frac{b+\sqrt{\Delta}}{2a} \right)=0$

$x=\frac{-b+\sqrt{\Delta}}{2a}$ ou $x=\frac{-b-\sqrt{\Delta}}{2a}$

On note $x_{1}=\frac{-b-\sqrt{\Delta}}{2a}$, $x_{2}=\frac{-b+\sqrt{\Delta}}{2a}$

Ainsi, $S=\left\lbrace \frac{-b-\sqrt{\Delta}}{2a}, \frac{-b+\sqrt{\Delta}}{2a} \right\rbrace $
\item Si $\Delta < 0$ alors Pas de factorisatrion de $S=\emptyset$
\item Si $\Delta = 0$ alors 

$\left( x+\frac{b}{2a}\right)^{2}=0$

$\Leftrightarrow \left( x+\frac{b}{2a}\right) \left( x+\frac{b}{2a}\right)=0$ 

$\Leftrightarrow x=-\frac{b}{2a}$ ou $x=-\frac{b}{2a}$.
l'équation admet une solution double notée $x_{0}=-\frac{b}{2a}$

$ S = \left\lbrace -\frac{b}{2a} \right\rbrace  $
\end{itemize}
$\ast$\underline{\textbf{\textcolor{red}{Exemple}}}\\
\subsection*{\underline{\textbf{\textcolor{red}{Théorème2}}}}
Soit l'équation $ax^{2}+bx+c=0$ et $b=2b'$ donc $\Delta'=b'^{2}-ac$
$\Delta'$ est appelé discriminant réduit.
\begin{itemize}
    \item Si $Delta' >0$ alors on aura 2 racines distinctes $x_{1}$ et $x_{2}$ tels que:
            \[x_{1}=\frac{-b'-\sqrt{\Delta'}}{a}\] et \[x_{2}=\frac{-b'+\sqrt{\Delta'}}{a}\]
            Ainsi $S=\left\lbrace \frac{-b'-\sqrt{\Delta'}}{a}, \frac{-b'+\sqrt{\Delta'}}{a} \right\rbrace$
        \end{itemize}
$\ast$\underline{\textbf{\textcolor{red}{Exemple}}}\\
\subsection*{\underline{\textbf{\textcolor{red}{Théorème3}}}}
$\ast$\underline{\textbf{\textcolor{red}{Exemple}}}\\
\subsection*{\underline{\textbf{\textcolor{red}{2.Somme et Produit des Racines}}}}
Soit $x_{1}$;$x_{2}$ les racines de l'équation $ax^{2}+bx+c=0$\\
$\ast$\underline{\textbf{\textcolor{red}{Somme}}}\\
$\ast$\underline{\textbf{\textcolor{red}{Exemple}}}\\
$\ast$\underline{\textbf{\textcolor{red}{Produit}}}\\
$\ast$\underline{\textbf{\textcolor{red}{Exemple}}}\\
\subsection*{\underline{\textbf{\textcolor{red}{3.Equation se ramenant à une équation du second degré}}}}
$\ast$\underline{\textbf{\textcolor{red}{Equation du type $ax^{4}+bx^{2}+c=0$}}}\\
$\ast$\underline{\textbf{\textcolor{red}{Exemple}}}\\
$\ast$\underline{\textbf{\textcolor{red}{Equation du type $ax^{2}+b|x|+c=0$}}}\\
$\ast$\underline{\textbf{\textcolor{red}{Exemple}}}\\
\section*{\underline{\textbf{\textcolor{red}{III.Inequation du second degré}}}}
\subsection*{\underline{\textbf{\textcolor{red}{Théorème3}}}}
\end{document}