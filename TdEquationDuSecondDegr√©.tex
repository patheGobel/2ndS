\documentclass[12pt]{article}
\usepackage{stmaryrd}
\usepackage{graphicx}
\usepackage[utf8]{inputenc}

\usepackage[french]{babel}
\usepackage[T1]{fontenc}
\usepackage{hyperref}
\usepackage{verbatim}

\usepackage{color, soul}

\usepackage{pgfplots}
\pgfplotsset{compat=1.15}
\usepackage{mathrsfs}

\usepackage{amsmath}
\usepackage{amsfonts}
\usepackage{amssymb}
\usepackage{tkz-tab}
\author{\\Lycée de Dindéfelo\\Mr BA}
\title{\textbf{Td Equation Du 2nd Degré}}
\date{\today}
\usepackage{tikz}
\usetikzlibrary{arrows, shapes.geometric, fit}

% Commande pour la couleur d'accentuation
\newcommand{\myul}[2][black]{\setulcolor{#1}\ul{#2}\setulcolor{black}}
\newcommand\tab[1][1cm]{\hspace*{#1}}

\begin{document}
\maketitle
\newpage
\section*{\underline{\textbf{\textcolor{red}{Exercice 1}}}}
$\ast$\textbf{Mettre sous la forme canonique les trinôme suivant :} \\
$A(x)=x^{2}+2x-3$\quad ; $B(x) = 5x^{2}-8x+3$ ; $C(x) =x^{2}-6x+8$ ; $D(x) =x^{2}-x+4$ ; 
$E(x) = 7x^{2}-11x + 13$ ; $F(x) = 2x^{2}+9x+1$ ; $G(x) =\sqrt{2}x^{2}+3\sqrt{3}x-1$ ; 
$H(x) =x^{2}-\sqrt{5}$ ; $I =x^{2} + 2mx-3m^{2}$ ;$J(x) =mx^{2}-(m^{2}-1)x-m$ ; 
$K(x) =(m-2)x^{2}+(5m-14)x-20$\\
$\ast$\textbf{Factoriser, si possible, chacun des trinômes suivants :} \\
$A(x)=-x^{2}+3x+2$ ; $B(x)=x^{2}+8x-15$ ; $C(x)=-15x^{2}+7x+8$ ; $D(x)=3x^{2}+x+1$ ; $E(x)=x^{2}-5x$ ; $F(x) =x^{2}+9x+49$ ; $G(x) =-\frac{1}{2}x^{2}+x-2$ ; $H(x) =x^{2}-3$ ;
$I(x) =2x^{2}-5x$ ; $J(x) =x^{2}-x\sqrt{2}+2$ ; $K(x) =-x^{2}+mx+1$ ;\\ $L(x) =x^{2}-x+2m$ ; $M(x)=(m-3)x^{2}-9x$ ; $N(x) =(m+3)x^{2}-2m-3$ ;
$O(x)=x^{2}-mx +\frac{1}{4}m^{2}+m+\frac{3}{4}$ ; $P(x)= x^{2} + 2(\sqrt{2}-2)X+5-4\sqrt{2}$ ; Q(x)=$4x^{2}-4\sqrt{2+3\sqrt{2}}x+2+3\sqrt{2}$
\section*{\underline{\textbf{\textcolor{red}{Exercice 2}}}}
Résoudre dans $\mathbb{R}$ les équations suivantes :\\
a°)$x^{2}-12x+36=0$ ; b°)$4x^{2}+12x=-9$ ; c°)$2x^{2}-5x-7=0$ ;\\ d°) $x^{2}-5x+6=0$ ; e°) $3^{2}-x=4$ ; 
$3x^{2}-10x+3=0$ ; g°) $-x^{2}-6x+16=0$ ; $h°)2x^{2}+5x+12=0$ ; $i°)-x^{2}+3x+4=0$ ; j°) $x^{2}-22x+105 = 0$ ;\\ k°) $5x^{2}+7x-34=0$ ; l°) $2x^{2}-x+3=0$ ; 
m°) $\frac{x^{2}}{2}-3x+\frac{5}{2}=0$ ;\\ n°) $x^{2}-(1+\sqrt{2})x+7+2\sqrt{10}=0$ ;
o°) $x^{2}+2(\sqrt{2}-2)x+5-4\sqrt{2}=0$ ;\\p°) $4x^{2}-4\sqrt{2+3\sqrt{2}}x+2+3\sqrt{2}=0$ ; q°)$(\sqrt{3}+1)x^{2} +2\sqrt{3}x+\sqrt{3}-1=0$
\section*{\underline{\textbf{\textcolor{red}{Exercice 3}}}}
A. Résoudre dans $\mathbb{R}$ les équations suivantes :\\
\textcolor{red}{$x^{2}-2|x|-3 = 0$} ; \textcolor{blue}{$x^{2}+x+2 = |x-5|$} ;
\textcolor{green}{$2x|x-1| = -|2x-3|$} ; \textcolor{red}{$x|x| = 6 - 5|x|$};\\
\textcolor{blue}{$3x^{2}-7|x|+4 = 0$};\textcolor{green}{$x^{4} + x^{2} = 6$};\textcolor{blue}{$x^{4}-11x^{2}+18 = 0$}  ;\textcolor{red}{$3x-5\sqrt{x}-2 = 0$};\\
\textcolor{green}{$(x+\frac{1}{x})^{2}-3(x+\frac{1}{x})+2=0$};\textcolor{red}{$(\frac{2x+1}{x-3})^{2}+2(\frac{2x+1}{x-3})=3$};\textcolor{blue}{$(\frac{1}{x^{2}+x+1})^{2}-(\frac{3}{x^{2}+x+1})=-2$};\\ \textcolor{green}{$(x^{2}+4x-3)^{2}=0$} ;
\textcolor{blue}{$(6x^{2}-3x-1)^{2}-(x^{2}+4x-3)^{2}=0$};\\ \textcolor{red}{$(-3x^{2}-5x+6)(-x^{2}+3x+4)=0$} ; \textcolor{green}{$[E(x)]^{2}-2E(x)-8 = 0$}\\
B. Déterminer, s’ils existent, les nombres x et y dont on connaît la somme S et
le produit P.\\ a°) S = 1 et P = -6 ; b°) S = 2 et P =-1 ; c°) S =-3 et P = 9\\
\section*{\underline{\textbf{\textcolor{red}{Exercice 4}}}}
L’équation $ax^{2}+bx+c=0$ est supposée avoir deux racines $x_{1}$ et $x_{2}$ . Calculer en
fonction de la somme et du produit des racines les expressions suivantes :\\
$\frac{1}{x_{1}^{2}}+\frac{1}{x_{2}^{2}}$ ; $\frac{1}{x_{1}^{3}}+\frac{1}{x_{2}^{3}}$ ; 
$\frac{x_{1}+3}{x_{2}+1}+\frac{x_{2}+3}{x_{1}+1}$ ; $(x_{1}+3)^{3}+(x_{2}+3)^{3}$
\section*{\underline{\textbf{\textcolor{red}{Exercice 5}}}}
Résoudre dans $\mathbb{R}$ les systèmes suivants :\\
\begin{equation*}
\begin{cases}
x + y = -7 \\
xy = \frac{49}{4}
\end{cases}
\begin{cases}
x^{2} + y^{2} = 10 \\
x+y = -2
\end{cases}
\begin{cases}
x^{2} - y^{2} = 5 \\
xy = 6
\end{cases}
\begin{cases}
x-y-1 = 3 \\
(x-2)(1-y) = 2
\end{cases}
\begin{cases}
|x| + |y| = 11 \\
|xy| = 30
\end{cases}
\end{equation*}

\begin{equation*}
\begin{cases}
\frac{1}{x} + \frac{1}{y} = 5 \\
xy = \frac{1}{6}
\end{cases}
\begin{cases}
\frac{2}{x} + \frac{5}{y} = -3 \\
\frac{1}{xy}=-1
\end{cases}
\begin{cases}
\frac{x}{y} + \frac{y}{x} = 2 \\
x+y = -2\sqrt{2}
\end{cases}
\begin{cases}
x^{3}+y^{3}= 7 \\
x^{2}+xy = -1
\end{cases}
\begin{cases}
\sqrt{x} + \sqrt{y} = 5 \\
xy = 36
\end{cases}
\end{equation*}

\begin{equation*}
\begin{cases}
x+y = 10 \\
\sqrt{xy}= 21
\end{cases}
\begin{cases}
(x+2)^{2}+y^{2}-y = 20 \\
xy = -10
\end{cases}
\begin{cases}
(x^{2}+x+1)^{2}+(y^{2}-y+1)^{2} = 13 \\
(x^{2}+x+1)(y^{2}-y+1) = 6
\end{cases}
\end{equation*}
\begin{equation*}
\begin{cases}
(x+y)(x-y)=2 \\
(x+y)^{2}+(x-y)^{2} = 3
\end{cases}
\begin{cases}
x+y = 5 \\
\sqrt{x}+\sqrt{y} = 3
\end{cases}
\end{equation*}
\section*{\underline{\textbf{\textcolor{red}{Exercice 6}}}}
On considère l’équation (E) : $(b^{2}+c^{2})x^{2}-2acx+a^{2}-b^{2}=0$ dans laquelle a, b et c
désignent trois nombres strictement positifs et x un réel.\\
1) A quelle condition doivent satisfaire a, b et c pour que l’équation (E) admette
des racines ?
Donner une interprétation géométrique de cette condition, en supposant que
a , b et c sont les mesures des côtés d’un triangle.\\
2) En supposant que l’équation ait des racines, montrer, en utilisant un
groupement judicieux des termes du premier membre, que ces racines sont
comprise entre -1 et 1.
\section*{\underline{\textbf{\textcolor{red}{Exercice 7}}}}
A. Soit m un paramètre réel. On considère le trinôme du second degré
$f(x) = x^{2} - 2(m - 1)x + m^{2} - 4$\\
1) Déterminer les valeurs de m pour les quelles le trinôme $f(x)$ admet deux
solutions distinctes.\\
2) On suppose dans cette question que $m \in]-\infty;\frac{5}{2}[.$\\
Déterminer suivant les valeurs de m le signe des racines de $f(x).$\\
B. Déterminer les valeurs de m pour lesquelles l’équation :\\
$(m - 2)x^{2} - 2(m - 1)x + m - 3 = 0$
admet des racines $x_{1}$ et $x_{2}$
\section*{\underline{\textbf{\textcolor{red}{Exercice 8}}}}
1) Déterminer les valeurs du paramètre réel \( m \) pour lesquelles l'équation 
(E1):\( (m - 1)x^2 + (m - 5)x + m - 2 = 0 \) admet deux solutions distinctes non nulles de signes contraires.

2) Pour quelles valeurs du paramètre réel \( m \), l'équation\\ 
(E2):\( (m - 4)x^2 + (m + 2)x - m = 0 \) admet-elle deux racines distinctes strictement positives ?

3) Pour quelles valeurs du paramètre \( m \), l'équation\\ 
(E3):\( (m + 1)x^2 + (2m + 3)x + m + 2 = 0 \) admet-elle deux solutions distinctes strictement négatives ?
\section*{\underline{\textbf{\textcolor{red}{Exercice 9}}}}
Soit l'équation \( (E) : x^2 - (2m + 1)x + m^2 - 2 = 0 \) où \( m \) est un paramètre réel.

1) Discuter suivant les valeurs de \( m \) l'existence et le nombre de solutions de 
\( (E) \).

2) Dans le cas où \( (E) \) admet deux solutions \( x_1 \) et \( x_2 \), déterminer \( m \) tels que :

a) \( x_1^2 + x_2^2 = 25 \) \\
b) \( |x_1 - x_2| = 5 \)

3) Dans le cas où \( (E) \) admet deux racines distinctes, calculer leur somme et leur produit et donner une relation indépendante de \( m \) liant ces racines.

4) Déterminer \( m \) pour que \( (E) \) admette :

a) Deux solutions de signes contraires. \\
b) Deux solutions positives. \\
c) Deux solutions négatives.
\section*{\underline{\textbf{\textcolor{red}{Exercice 10}}}}
Résoudre dans $\mathbb{R}$, les inéquations suivantes :\\
\textcolor{blue}{$3x^{2}-4x+1>0$} ; \textcolor{red}{$-2x^{2}-x+1\geq 0$};
\textcolor{green}{$-9x^{2}+12x-4\leq 0$} ; \textcolor{blue}{$2x^{2}-4\sqrt{3}x+6\leq 0$} ; 
\textcolor{red}{$-16x^{2}+24x-9 \geq 0$} ;\textcolor{green}{$-3x^{2}+75<0$} ;
\textcolor{blue}{$x^{2}+2(\sqrt{2}-2)x+5-4\sqrt{2}>0$};
\textcolor{red}{$(-9x^{2}+12x-4)(2x-3)\leq 0$};
\textcolor{green}{$2x^{4}-5x^{2}+3>0$};\textcolor{blue}{$(2x+1)(x-2)-4x^{2}+1\leq 0$};\\
\textcolor{red}{$(-2x^{2}+3x+6)(5x^{2}-3x-14)\leq 0$} ;
\begin{equation*}
\begin{cases}
3x^{2}-4x+1>0 \\
-9x^{2}+12x-4\leq 0
\end{cases}
\frac{2x^{2}-3x-2}{-x^{2}-x+2}\geq 0 ; 
\frac{3x^{2}+8x-11}{2x^{2}+5x-7}\geq 1 ;
2<\frac{x-3}{x-5}\leq 3
\end{equation*}
\section*{\underline{\textbf{\textcolor{red}{Exercice 11}}}}
A. Résoudre les systèmes suivant par la méthode de Cramer.\\
\begin{equation*}
\begin{cases}
-x+y=3 \\
2x+3y = 4
\end{cases}
\begin{cases}
x-2y = 1 \\
3x+y = 10
\end{cases}
\begin{cases}
2x+4y = 2 \\
3x+6y = -11
\end{cases}
\begin{cases}
x+y = 2 \\
x+2y = -1
\end{cases}
\begin{cases}
x\sqrt{2}+y = 1 \\
x-3y\sqrt{2} = 4\sqrt{2}
\end{cases}
\end{equation*}
\begin{equation*}
\begin{cases}
x-3y=4\\
-2x+6y=7\\
-x+3y=1
\end{cases}
\begin{cases}
3\sqrt{x}+4\sqrt{x}=1\\
5\sqrt{x}+2\sqrt{x}=3
\end{cases}
\begin{cases}
\frac{1}{x}+\frac{2}{y}\\
\frac{3}{x}-\frac{1}{y}=4
\end{cases}
\begin{cases}
x^{2}-2x+2(y^{2}+4)+3=0\\
3(x^{2}-2x)-(y^{2}+4)=12
\end{cases}
\end{equation*}
\begin{equation*}
\begin{cases}
\frac{x}{2}+y=\frac{3}{4} \\
x+2y = \frac{3}{2}
\end{cases}
\begin{cases}
\frac{1}{x}+\frac{2}{y} = -1 \\
\frac{3}{x}-\frac{1}{y} = 4 
\end{cases}
\begin{cases}
|x|-11|y| = -3 \\
|x|+7|y| = 15
\end{cases}
\begin{cases}
\sqrt{x}+2\sqrt{y}= 2\sqrt{2} \\
3\sqrt{x}+\sqrt{2y}= 2
\end{cases}
\begin{cases}
\frac{1}{x-1}+\frac{2}{y+1} = -1 \\
\frac{3}{x-1}-\frac{1}{y+1} = 5
\end{cases}
\end{equation*}
B. Résoudre les systèmes d’inéquations suivants\\
\begin{equation*}
\begin{cases}
x\leq-1\\
x-y>2
\end{cases}
\begin{cases}
x\leq-1\\
x-y>2
\end{cases}
\begin{cases}
x-2y>3\\
-3x+y\leq 0
\end{cases}
\begin{cases}
x \geq 3\\
	y<-1\\
\frac{x}{2} \leq y
\end{cases}
\begin{cases}
5x + 3y \leq 1\\
-x+2y\geq 1 \\
2x-y\geq -2
\end{cases}
\begin{cases}
x>y\\
x+y < -1\\
3x \geq 2y +1
\end{cases}
\end{equation*}
C.Resoudre graphiquement les systèmes suivantes
\begin{equation*}
\begin{cases}
2x-y\leq 3\\
1 \leq 2x-y
\end{cases}
\begin{cases}
x+y-3<0\\
y+2x>\\
y\leq 2
\end{cases}
\begin{cases}
2x+y\geq 2\\
x+2y\leq 4\\
x+2y\geq 6
\end{cases}
\end{equation*}
\section*{\underline{\textbf{\textcolor{red}{Exercice 12}}}}
On considère l’équation a coefficient symétrique suivant :\\
$(E):ax^{4}+bx^{3}+cx^{2}+bx+c=0$ , où a , b et c sont trois nombre réels tels que $a \neq 0$\\
1) Montrer que 0 n’est pas une solution de l’équation (E).\\
2) Montre que l’équation (E) est équivalent à l’équation\\
(E'):$a(x^{2}+\frac{1}{x^{2}})+b(x+\frac{1}{x})+c$\\
3) On pose $X=x+\frac{1}{x}$\\
a)Montrer que $x^{2}+\frac{1}{x^{2}}=X^{2}-2$\\
b) Montrer que l’équation (E) est équivalente à :
\begin{equation*}
\begin{cases}
x + y = -7 \\
xy = \frac{49}{4}
\end{cases}
\end{equation*}
c) Résoudre dans $\mathbb{R}$ les équations suivants :\\
$x^{4}-3x^{3} + 4x^{2}-3x + 1 = 0$ ; $-x^{4} + 2x^{3} -x^{2} + 2x - 1 = 0$ ; 
$x^{4} + x^{3} - 4x^{2} + x + 1 = 0$ 
\section*{\underline{\textbf{\textcolor{red}{Problème1}}}}
Une entreprise fabrique un produit chimique dont le coût total journalier de production pour \( x \) litres est donné par \( C(x) = 0.5x^2 + 2x + 200 \) pour tout \( x \) dans l'intervalle \( [1, 50] \), le coût étant exprimé en milliers de dirhams. Le prix de vente d'un litre de ce produit chimique est de 23000 dirhams. Soit \( B(x) \) le bénéfice de l'entreprise.

1. Montrer que \( B(x) = -0.5x^2 + 21x - 200 \).

a. Résoudre l'inéquation \( B(x) \geq 0 \).

b. Interpréter le résultat obtenu.

2. Déterminer la quantité à produire pour avoir un bénéfice de 20500 dirhams.

3. L'entreprise peut-elle réaliser un bénéfice de plus de 20500 ?

\end{document}